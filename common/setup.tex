
%%% Вывод типов ссылок в библиографии %%%
\makeatletter
\@ifundefined{c@mediadisplay}{
	\newcounter{mediadisplay}
	\setcounter{mediadisplay}{0}
	% 0 --- не делать ничего; надписи [Текст] и [Эл. ресурс] будут выводиться только в ссылках с заполненным полем `media`;
	% 1 --- автоматически добавлять надпись [Текст] к ссылкам с незаполненным полем `media`; таким образом, у всех источников будет указан тип, что соответствует требованиям ГОСТ
	% 2 --- автоматически удалять надписи [Текст], [Эл. Ресурс] и др.; не соответствует ГОСТ
	% 3 --- автоматически удалять надпись [Текст]; не соответствует ГОСТ
	% 4 --- автоматически удалять надпись [Эл. Ресурс]; не соответствует ГОСТ
}{}
\makeatother


% %%% INSTRUCTIONS %%%
% 1) Ссылки с архива, как и любого другого эл. ресурса, нужно будет определить как @online вместо дефолтного @article из школяра.
% 2) Обязательно нужно добавить два поля: url и urldate, чтобы появились требуемые ГОСТом адрес электронного ресурса и дата последнего посещения. Пример:
%		url={https://arxiv.org/abs/2502.17029},
% 		urldate={2025-09-21},
% 3) Поле типа "journal={arXiv preprint arXiv:1412.6980}" можно удалить, так как эта информация более не нужна. Опционально можно перенести содержимое в поле howpublished, но я так не делал.
% 4) Обязательно нужно было добавлять поле "media={eresource}", чтобы появилась требуемая ГОСТом пометка "[Electronic Resource]", но я добавил ниже код в \DeclareSourcemap{...}, который добавляет это поле в каждый @online автоматически.
