%%% Шаблон %%%
% Абзацный отступ. Должен быть одинаковым по всему тексту и равен пяти знакам (ГОСТ Р 7.0.11-2011, 5.3.7).
\AtBeginDocument{\setlength{\parindent}{2.5em}}


%%% Таблицы %%%
\DeclareCaptionLabelSeparator{tabsep}{\tablabelsep} % нумерация таблиц
%\DeclareCaptionFormat{split}{\splitformatlabel#1\par\splitformattext#3}

\captionsetup[table]{
	format=\tabformat,                % формат подписи (plain|hang)
	font=normal,                      % нормальные размер, цвет, стиль шрифта
	skip=0pt,                         % отбивка под подписью
	parskip=0pt,                      % отбивка между параграфами подписи
	position=above,                   % положение подписи
	justification=\tabjust,           % центровка
	indent=\tabindent,                % смещение строк после первой
	labelsep=tabsep,                  % разделитель
	singlelinecheck=\tabsinglecenter, % не выравнивать по центру, если умещается в одну строку
}


%%% Рисунки %%%
\DeclareCaptionLabelSeparator{figsep}{\figlabelsep} % нумерация рисунков

\captionsetup[figure]{
	format=plain,                     % формат подписи (plain|hang)
	font=normal,                      % нормальные размер, цвет, стиль шрифта
	skip=.0pt,                        % отбивка под подписью %%% skip=6pt, чтобы подпись не прилипала к рисунку.
	parskip=.0pt,                     % отбивка между параграфами подписи
	position=below,                   % положение подписи
	singlelinecheck=true,             % выравнивание по центру, если умещается в одну строку
	justification=\tabjust,
	%justification=centerlast,		  % в ГОСТе нет требования - так что пофиг.
	labelsep=figsep,                  % разделитель
}


%%% Настройки гиперссылок %%%
\hypersetup{
	linktocpage=true,           % ссылки с номера страницы в оглавлении, списке таблиц и списке рисунков
	%    linktoc=all,                % both the section and page part are links
	%    pdfpagelabels=false,        % set PDF page labels (true|false)
	plainpages=false,           % Forces page anchors to be named by the Arabic form  of the page number, rather than the formatted form
	colorlinks,                 % ссылки отображаются раскрашенным текстом, а не раскрашенным прямоугольником, вокруг текста
	linkcolor={linkcolor},      % цвет ссылок типа ref, eqref и подобных
	citecolor={citecolor},      % цвет ссылок-цитат
	urlcolor={urlcolor},        % цвет гиперссылок
	%    hidelinks,                  % Hide links (removing color and border)
	%    pdftitle={\thesisTitle},    % Заголовок
	pdftitle={\thesisTitlePDF},    % Заголовок
	pdfauthor={\thesisAuthor},  % Автор
	pdfsubject={\thesisSpecialtyNumber\ \thesisSpecialtyTitle},      % Тема
	%    pdfcreator={Создатель},     % Создатель, Приложение
	%    pdfproducer={Производитель},% Производитель, Производитель PDF
	pdfkeywords={\keywords},    % Ключевые слова
	pdflang={ru},
}


%%% Списки %%%
% Используем короткое тире (endash) для ненумерованных списков 
\renewcommand{\labelitemi}{\normalfont -} % ГОСТ 2.105-95, пункт 4.1.7, требует дефиса
%\renewcommand{\labelitemi}{\normalfont\bfseries{--}} % а так лучше смотрится

%%% Списки по ГОСТ (адаптация для англ. текста) %%%
% ГОСТ 2.105–95, п. 4.1.7: перечисления строчными буквами.
% В англ. диссертации — используем латиницу вместо кириллицы.
% 1-й уровень — латинские буквы: a), b), c) ...
%\renewcommand{\theenumi}{\alph{enumi}}
%\renewcommand{\labelenumi}{\theenumi)}
% 2-й уровень — арабские цифры: 1), 2), 3) ...
%\renewcommand{\theenumii}{\arabic{enumii}}
%\renewcommand{\labelenumii}{\theenumii)}
% 3-й уровень — латинские буквы: a), b), c) ...
% (на случай глубокой вложенности, хотя обычно не используется)
%\renewcommand{\theenumiii}{\alph{enumiii}}
%\renewcommand{\labelenumiii}{\theenumiii)}

\setlist{nosep,%                                    % Единый стиль для всех списков (пакет enumitem), без дополнительных интервалов.
	labelindent=\parindent,leftmargin=*%            % Каждый пункт, подпункт и перечисление записывают с абзацного отступа (ГОСТ 2.105-95, 4.1.8)
}


%%% Для параграфа Dissertation structure %%%
%%http://www.linux.org.ru/forum/general/6993203#comment-6994589 (используется totcount)
\makeatletter
\def\formtotal#1#2#3#4#5{%
	\newcount\@c
	\@c\totvalue{#1}\relax
	\newcount\@last
	\newcount\@pnul
	\@last\@c\relax
	\divide\@last 10
	\@pnul\@last\relax
	\divide\@pnul 10
	\multiply\@pnul-10
	\advance\@pnul\@last
	\multiply\@last-10
	\advance\@last\@c
	#2%
	\ifnum\@pnul=1#5\else%
	\ifcase\@last#5\or#3\or#4\or#4\or#4\else#5\fi
	\fi
}
\makeatother

\newcommand{\formbytotal}[5]{\total{#1}~\formtotal{#1}{#2}{#3}{#4}{#5}}

