

%\DeclareSourcemap{
	%	\maps[overwrite]{
		%		\map{
			%			\pertype{online}
			%			\step[fieldset=media, fieldvalue=eresource]
			%		}
		%	}
	%}

%http://tex.stackexchange.com/a/141831/79756
%There is a way to automatically map the language field to the langid field. The following lines in the preamble should be enough to do that.
%This command will copy the language field into the langid field and will then delete the contents of the language field. The language field will only be deleted if it was successfully copied into the langid field.
\DeclareSourcemap{ %модификация bib файла перед тем, как им займётся biblatex
	\maps{
		\map{% перекидываем значения полей language в поля langid, которыми пользуется biblatex
			\step[fieldsource=language, fieldset=langid, origfieldval, final]
			\step[fieldset=language, null]
		}
		\map{% перекидываем значения полей numpages в поля pagetotal, которыми пользуется biblatex
			\step[fieldsource=numpages, fieldset=pagetotal, origfieldval, final]
			\step[fieldset=numpages, null]
		}
		\map{% перекидываем значения полей pagestotal в поля pagetotal, которыми пользуется biblatex
			\step[fieldsource=pagestotal, fieldset=pagetotal, origfieldval, final]
			\step[fieldset=pagestotal, null]
		}
		\map[overwrite]{% перекидываем значения полей shortjournal, если они есть, в поля journal, которыми пользуется biblatex
			\step[fieldsource=shortjournal, final]
			\step[fieldset=journal, origfieldval]
			\step[fieldset=shortjournal, null]
		}
		\map[overwrite]{% перекидываем значения полей shortbooktitle, если они есть, в поля booktitle, которыми пользуется biblatex
			\step[fieldsource=shortbooktitle, final]
			\step[fieldset=booktitle, origfieldval]
			\step[fieldset=shortbooktitle, null]
		}
		%		\map{% если в поле medium написано "Электронный ресурс", то устанавливаем поле media, которым пользуется biblatex, в значение eresource.
			%			\step[fieldsource=medium,
			%			match=\regexp{Электронный\s+ресурс},
			%			final]
			%			\step[fieldset=media, fieldvalue=eresource]
			%			\step[fieldset=medium, null]
			%		}
		\map{%
			\pertype{online}
			\step[fieldset=media, fieldvalue=eresource]
		}
		\map[overwrite]{% стираем значения всех полей issn
			\step[fieldset=issn, null]
		}
		\map[overwrite]{% стираем значения всех полей abstract, поскольку ими не пользуемся, а там бывают "неприятные" латеху символы
			\step[fieldsource=abstract]
			\step[fieldset=abstract,null]
		}
		\map[overwrite]{ % переделка формата записи даты
			\step[fieldsource=urldate,
			match=\regexp{([0-9]{2})\.([0-9]{2})\.([0-9]{4})},
			replace={$3-$2-$1$4}, % $4 вставлен исключительно ради нормальной работы программ подсветки синтаксиса, которые некорректно обрабатывают $ в таких конструкциях
			final]
		}
		\map[overwrite]{ % стираем ключевые слова
			\step[fieldsource=keywords]
			\step[fieldset=keywords,null]
		}
		% реализация foreach различается для biblatex v3.12 и v3.13.
		% Для версии v3.13 эта конструкция заменяет последующие 7 структур map
		\map[overwrite,foreach={authorvak,authorscopus,authorwos,authorconf,authorother,authorpatent,authorprogram}]{ % записываем информацию о типе публикации в ключевые слова
			\step[fieldsource=$MAPLOOP,final=true]
			\step[fieldset=keywords,fieldvalue={,biblio$MAPLOOP},append=true]
		}
		\map[overwrite]{ % добавляем ключевые слова, чтобы различать источники
			\perdatasource{biblio/external.bib}
			\step[fieldset=keywords, fieldvalue={,biblioexternal},append=true]
		}
		\map[overwrite]{ % добавляем ключевые слова, чтобы различать источники
			\perdatasource{biblio/author.bib}
			\step[fieldset=keywords, fieldvalue={,biblioauthor},append=true]
		}
		\map[overwrite]{ % добавляем ключевые слова, чтобы различать источники
			\perdatasource{biblio/registered.bib}
			\step[fieldset=keywords, fieldvalue={,biblioregistered},append=true]
		}
		\map[overwrite]{ % добавляем ключевые слова, чтобы различать источники
			\step[fieldset=keywords, fieldvalue={,bibliofull},append=true]
		}
		%        \map[overwrite]{% стираем значения всех полей series
			%            \step[fieldset=series, null]
			%        }
		\map[overwrite]{% перекидываем значения полей howpublished в поля organization для типа online
			\step[typesource=online, typetarget=online, final]
			\step[fieldsource=howpublished, fieldset=organization, origfieldval]
			\step[fieldset=howpublished, null]
		}
	}
}

\ifnumequal{\value{mediadisplay}}{1}{
	\DeclareSourcemap{
		\maps{%
			\map{% использование media=text по умолчанию
				\step[fieldset=media, fieldvalue=text]
			}
		}
	}
}{}
\ifnumequal{\value{mediadisplay}}{2}{
	\DeclareSourcemap{
		\maps{%
			\map[overwrite]{% удаление всех записей media
				\step[fieldset=media, null]
			}
		}
	}
}{}
\ifnumequal{\value{mediadisplay}}{3}{
	\DeclareSourcemap{
		\maps{
			\map[overwrite]{% стираем значения всех полей media=text
				\step[fieldsource=media,match={text},final]
				\step[fieldset=media, null]
			}
		}
	}
}{}
\ifnumequal{\value{mediadisplay}}{4}{
	\DeclareSourcemap{
		\maps{
			\map[overwrite]{% стираем значения всех полей media=eresource
				\step[fieldsource=media,match={eresource},final]
				\step[fieldset=media, null]
			}
		}
	}
}{}

\ifsynopsis
\else
\DeclareSourcemap{ %модификация bib файла перед тем, как им займётся biblatex
	\maps{
		\map[overwrite]{% стираем значения всех полей addendum
			\perdatasource{biblio/author.bib}
			\step[fieldset=addendum, null] %чтобы избавиться от информации об объёме авторских статей, в отличие от автореферата
		}
	}
}
\fi