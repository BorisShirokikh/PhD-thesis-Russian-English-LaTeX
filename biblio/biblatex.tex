%%% Библиография через biblatex + biblatex-gost, движок biber %%%
\usepackage{csquotes} % рекомендуется для biblatex

\usepackage[
backend=biber,         % движок
bibencoding=utf8,      % кодировка bib-файла
sorting=none,          % порядок литературы как в bib-файле
style=gost-numeric,    % стиль ГОСТ
language=autobib,      % язык цитат подстраивается под babel/polyglossia
autolang=other,        % многоязычная библиография
clearlang=true,         % сброс поля language если совпадает с основным языком
defernumbers=true,     % номера в списке литературы после всех цитирований
sortcites=true,        % сортировка нескольких цитат в скобках
doi=false,             % отключаем DOI
isbn=false             % отключаем ISBN
]{biblatex}


\ExecuteBibliographyOptions{defernumbers=true,refsection=none}


%%% Подключение файлов bib %%%
%\addbibresource[label=bl-external]{biblio/external_rebiber.bib}
\addbibresource[label=bl-external]{biblio/external.bib}
\addbibresource[label=bl-author]{biblio/author.bib}
\addbibresource[label=bl-registered]{biblio/registered.bib}




%\DeclareSourcemap{
	%	\maps[overwrite]{
		%		\map{
			%			\pertype{online}
			%			\step[fieldset=media, fieldvalue=eresource]
			%		}
		%	}
	%}

%http://tex.stackexchange.com/a/141831/79756
%There is a way to automatically map the language field to the langid field. The following lines in the preamble should be enough to do that.
%This command will copy the language field into the langid field and will then delete the contents of the language field. The language field will only be deleted if it was successfully copied into the langid field.
\DeclareSourcemap{ %модификация bib файла перед тем, как им займётся biblatex
	\maps{
		\map{% перекидываем значения полей language в поля langid, которыми пользуется biblatex
			\step[fieldsource=language, fieldset=langid, origfieldval, final]
			\step[fieldset=language, null]
		}
		\map{% перекидываем значения полей numpages в поля pagetotal, которыми пользуется biblatex
			\step[fieldsource=numpages, fieldset=pagetotal, origfieldval, final]
			\step[fieldset=numpages, null]
		}
		\map{% перекидываем значения полей pagestotal в поля pagetotal, которыми пользуется biblatex
			\step[fieldsource=pagestotal, fieldset=pagetotal, origfieldval, final]
			\step[fieldset=pagestotal, null]
		}
		\map[overwrite]{% перекидываем значения полей shortjournal, если они есть, в поля journal, которыми пользуется biblatex
			\step[fieldsource=shortjournal, final]
			\step[fieldset=journal, origfieldval]
			\step[fieldset=shortjournal, null]
		}
		\map[overwrite]{% перекидываем значения полей shortbooktitle, если они есть, в поля booktitle, которыми пользуется biblatex
			\step[fieldsource=shortbooktitle, final]
			\step[fieldset=booktitle, origfieldval]
			\step[fieldset=shortbooktitle, null]
		}
		%		\map{% если в поле medium написано "Электронный ресурс", то устанавливаем поле media, которым пользуется biblatex, в значение eresource.
			%			\step[fieldsource=medium,
			%			match=\regexp{Электронный\s+ресурс},
			%			final]
			%			\step[fieldset=media, fieldvalue=eresource]
			%			\step[fieldset=medium, null]
			%		}
		\map{%
			\pertype{online}
			\step[fieldset=media, fieldvalue=eresource]
		}
		\map[overwrite]{% стираем значения всех полей issn
			\step[fieldset=issn, null]
		}
		\map[overwrite]{% стираем значения всех полей abstract, поскольку ими не пользуемся, а там бывают "неприятные" латеху символы
			\step[fieldsource=abstract]
			\step[fieldset=abstract,null]
		}
		\map[overwrite]{ % переделка формата записи даты
			\step[fieldsource=urldate,
			match=\regexp{([0-9]{2})\.([0-9]{2})\.([0-9]{4})},
			replace={$3-$2-$1$4}, % $4 вставлен исключительно ради нормальной работы программ подсветки синтаксиса, которые некорректно обрабатывают $ в таких конструкциях
			final]
		}
		\map[overwrite]{ % стираем ключевые слова
			\step[fieldsource=keywords]
			\step[fieldset=keywords,null]
		}
		% реализация foreach различается для biblatex v3.12 и v3.13.
		% Для версии v3.13 эта конструкция заменяет последующие 7 структур map
		\map[overwrite,foreach={authorvak,authorscopus,authorwos,authorconf,authorother,authorpatent,authorprogram}]{ % записываем информацию о типе публикации в ключевые слова
			\step[fieldsource=$MAPLOOP,final=true]
			\step[fieldset=keywords,fieldvalue={,biblio$MAPLOOP},append=true]
		}
		\map[overwrite]{ % добавляем ключевые слова, чтобы различать источники
			\perdatasource{biblio/external.bib}
			\step[fieldset=keywords, fieldvalue={,biblioexternal},append=true]
		}
		\map[overwrite]{ % добавляем ключевые слова, чтобы различать источники
			\perdatasource{biblio/author.bib}
			\step[fieldset=keywords, fieldvalue={,biblioauthor},append=true]
		}
		\map[overwrite]{ % добавляем ключевые слова, чтобы различать источники
			\perdatasource{biblio/registered.bib}
			\step[fieldset=keywords, fieldvalue={,biblioregistered},append=true]
		}
		\map[overwrite]{ % добавляем ключевые слова, чтобы различать источники
			\step[fieldset=keywords, fieldvalue={,bibliofull},append=true]
		}
		%        \map[overwrite]{% стираем значения всех полей series
			%            \step[fieldset=series, null]
			%        }
		\map[overwrite]{% перекидываем значения полей howpublished в поля organization для типа online
			\step[typesource=online, typetarget=online, final]
			\step[fieldsource=howpublished, fieldset=organization, origfieldval]
			\step[fieldset=howpublished, null]
		}
	}
}

\ifnumequal{\value{mediadisplay}}{1}{
	\DeclareSourcemap{
		\maps{%
			\map{% использование media=text по умолчанию
				\step[fieldset=media, fieldvalue=text]
			}
		}
	}
}{}
\ifnumequal{\value{mediadisplay}}{2}{
	\DeclareSourcemap{
		\maps{%
			\map[overwrite]{% удаление всех записей media
				\step[fieldset=media, null]
			}
		}
	}
}{}
\ifnumequal{\value{mediadisplay}}{3}{
	\DeclareSourcemap{
		\maps{
			\map[overwrite]{% стираем значения всех полей media=text
				\step[fieldsource=media,match={text},final]
				\step[fieldset=media, null]
			}
		}
	}
}{}
\ifnumequal{\value{mediadisplay}}{4}{
	\DeclareSourcemap{
		\maps{
			\map[overwrite]{% стираем значения всех полей media=eresource
				\step[fieldsource=media,match={eresource},final]
				\step[fieldset=media, null]
			}
		}
	}
}{}

\ifsynopsis
\else
\DeclareSourcemap{ %модификация bib файла перед тем, как им займётся biblatex
	\maps{
		\map[overwrite]{% стираем значения всех полей addendum
			\perdatasource{biblio/author.bib}
			\step[fieldset=addendum, null] %чтобы избавиться от информации об объёме авторских статей, в отличие от автореферата
		}
	}
}
\fi


%%% Тире как разделитель в библиографии традиционной руской длины:
\renewcommand*{\newblockpunct}{\addperiod\addnbspace\textemdash\space\bibsentence}


%%% В списке литературы обозначение одной буквой диапазона страниц англоязычного источника %%%
\DefineBibliographyStrings{english}{%
	pages = {p\adddot} %заглавность буквы затем по месту определяется работой самого biblatex
}


%%% Исправление длины тире в диапазонах %%%
% \cyrdash --- тире «русской» длины, \textendash --- en-dash
\DefineBibliographyExtras{russian}{%
	\protected\def\bibrangedash{%
		\textendash\penalty\value{abbrvpenalty}}% almost unbreakable dash
	\protected\def\bibdaterangesep{\bibrangedash}%тире для дат
}
\DefineBibliographyExtras{english}{%
	\protected\def\bibrangedash{%
		\textendash\penalty\value{abbrvpenalty}}% almost unbreakable dash
	\protected\def\bibdaterangesep{\bibrangedash}%тире для дат
}

%Set higher penalty for breaking in number, dates and pages ranges
\setcounter{abbrvpenalty}{10000} % default is \hyphenpenalty which is 12

%Set higher penalty for breaking in names
\setcounter{highnamepenalty}{10000} % If you prefer the traditional BibTeX behavior (no linebreaks at highnamepenalty breakpoints), set it to ‘infinite’ (10 000 or higher).
\setcounter{lownamepenalty}{10000}


%%% Макросы автоматического подсчёта количества авторских публикаций.
% Печатают невидимую (пустую) библиографию, считая количество источников.
% http://tex.stackexchange.com/a/66851/79756
%
\makeatletter
\newtotcounter{citenum}
\defbibenvironment{counter}
{\setcounter{citenum}{0}\renewcommand{\blx@driver}[1]{}} % begin code: убирает весь выводимый текст
{} % end code
{\stepcounter{citenum}} % item code: cчитает "печатаемые в библиографию" источники

\newtotcounter{citeauthorvak}
\defbibenvironment{countauthorvak}
{\setcounter{citeauthorvak}{0}\renewcommand{\blx@driver}[1]{}}
{}
{\stepcounter{citeauthorvak}}

\newtotcounter{citeauthorscopus}
\defbibenvironment{countauthorscopus}
{\setcounter{citeauthorscopus}{0}\renewcommand{\blx@driver}[1]{}}
{}
{\stepcounter{citeauthorscopus}}

\newtotcounter{citeauthorwos}
\defbibenvironment{countauthorwos}
{\setcounter{citeauthorwos}{0}\renewcommand{\blx@driver}[1]{}}
{}
{\stepcounter{citeauthorwos}}

\newtotcounter{citeauthorother}
\defbibenvironment{countauthorother}
{\setcounter{citeauthorother}{0}\renewcommand{\blx@driver}[1]{}}
{}
{\stepcounter{citeauthorother}}

\newtotcounter{citeauthorconf}
\defbibenvironment{countauthorconf}
{\setcounter{citeauthorconf}{0}\renewcommand{\blx@driver}[1]{}}
{}
{\stepcounter{citeauthorconf}}

\newtotcounter{citeauthor}
\defbibenvironment{countauthor}
{\setcounter{citeauthor}{0}\renewcommand{\blx@driver}[1]{}}
{}
{\stepcounter{citeauthor}}

\newtotcounter{citeauthorvakscopuswos}
\defbibenvironment{countauthorvakscopuswos}
{\setcounter{citeauthorvakscopuswos}{0}\renewcommand{\blx@driver}[1]{}}
{}
{\stepcounter{citeauthorvakscopuswos}}

\newtotcounter{citeauthorscopuswos}
\defbibenvironment{countauthorscopuswos}
{\setcounter{citeauthorscopuswos}{0}\renewcommand{\blx@driver}[1]{}}
{}
{\stepcounter{citeauthorscopuswos}}

\newtotcounter{citeregistered}
\defbibenvironment{countregistered}
{\setcounter{citeregistered}{0}\renewcommand{\blx@driver}[1]{}}
{}
{\stepcounter{citeregistered}}

\newtotcounter{citeauthorpatent}
\defbibenvironment{countauthorpatent}
{\setcounter{citeauthorpatent}{0}\renewcommand{\blx@driver}[1]{}}
{}
{\stepcounter{citeauthorpatent}}

\newtotcounter{citeauthorprogram}
\defbibenvironment{countauthorprogram}
{\setcounter{citeauthorprogram}{0}\renewcommand{\blx@driver}[1]{}}
{}
{\stepcounter{citeauthorprogram}}

\newtotcounter{citeexternal}
\defbibenvironment{countexternal}
{\setcounter{citeexternal}{0}\renewcommand{\blx@driver}[1]{}}
{}
{\stepcounter{citeexternal}}
\makeatother

\defbibheading{nobibheading}{} % пустой заголовок, для подсчёта публикаций с помощью невидимой библиографии
\defbibheading{pubgroup}{\section*{#1}} % обычный стиль, заголовок-секция
\defbibheading{pubsubgroup}{\noindent\textbf{#1}} % для подразделов "по типу источника"


%%% Команды для вывода списка литературы %%%
\newcommand*{\insertbibliofull}{\printbibliography[keyword=bibliofull]}
\newcommand*{\insertbiblioauthor}{\printbibliography[heading=pubgroup, keyword=biblioauthor, title=\bibtitleauthorEn]}
\newcommand*{\insertbiblioexternal}{\printbibliography[heading=pubgroup, keyword=biblioexternal, title=\bibtitlefullEn]}
\newcommand*{\insertbiblioregistered}{\printbibliography[heading=none, keyword=biblioregistered, title={}]}
%\newcommand*{\insertbiblioauthorimportant}{\printbibliography[heading=pubgroup, section=2, filter=papersregistered, title=\bibtitleauthorimportant]}

% Вариант вывода печатных работ автора, с группировкой по типу источника.
% Порядок команд `\printbibliography` должен соответствовать порядку в файле common/characteristic.tex
%\newcommand*{\insertbiblioauthorgrouped}{
	%	\section*{\bibtitleauthor}
	%	\ifsynopsis
	%	\printbibliography[heading=pubsubgroup, section=0, keyword=biblioauthorvak,    title=\bibtitleauthorvak,resetnumbers=true] % Работы автора из списка ВАК (сброс нумерации)
	%	\else
	%	\printbibliography[heading=pubsubgroup, section=0, keyword=biblioauthorvak,    title=\bibtitleauthorvak,resetnumbers=false] % Работы автора из списка ВАК (сквозная нумерация)
	%	\fi
	%	\printbibliography[heading=pubsubgroup, section=0, keyword=biblioauthorwos,    title=\bibtitleauthorwos,resetnumbers=false]% Работы автора, индексируемые Web of Science
	%	\printbibliography[heading=pubsubgroup, section=0, keyword=biblioauthorscopus, title=\bibtitleauthorscopus,resetnumbers=false]% Работы автора, индексируемые Scopus
	%	\printbibliography[heading=pubsubgroup, section=0, keyword=biblioauthorpatent, title=\bibtitleauthorpatent,resetnumbers=false]% Патенты
	%	\printbibliography[heading=pubsubgroup, section=0, keyword=biblioauthorprogram,title=\bibtitleauthorprogram,resetnumbers=false]% Программы для ЭВМ
	%	\printbibliography[heading=pubsubgroup, section=0, keyword=biblioauthorconf,   title=\bibtitleauthorconf,resetnumbers=false]% Тезисы конференций
	%	\printbibliography[heading=pubsubgroup, section=0, keyword=biblioauthorother,  title=\bibtitleauthorother,resetnumbers=false]% Прочие работы автора
	%}




\DeclareFieldFormat{number}{Is.\ #1}
% biblatex-gost defines \DeclareFieldFormat{number}{No.\ #1} by default.
% The line above resets it to print the ~raw number only~ Issue instead Number (GOST feature).
\DeclareFieldFormat{pages}{P.\ #1}
% For some reason, it does not print "P." in the case of a single page.


\newbibmacro*{my:urldate}{\printtext{(visited on \thefield{urlmonth}/\thefield{urlday}/\thefield{urlyear})}}


\newbibmacro*{my:core}{%
	% 1) Authors
	\printnames{author}%
	\addperiod\space%
	% 2) Title
	\printfield[titlecase]{title}%
	\setunit{\addspace//\space}%
	% 3) Journal
	\iffieldundef{journaltitle}%
		{\printfield[emph]{booktitle}}%
		{\printfield[emph]{journaltitle}}%
	\newunit\newblock%
	% 4) Year
	\printfield{year}%
	\newunit\newblock%
	% 5) Volume / Issue
	\iffieldundef{volume}{}{\printtext{\printfield{volume}}}%
	\iffieldundef{number}{}{\setunit{\addcomma\space}\printtext{\printfield{number}}}%
	\newunit\newblock%
	% 6) Pages
	\iffieldundef{pages}{}{\printtext{\printfield{pages}}}%
}


\newbibmacro*{my:online}{%
	% 1) Authors
	\printnames{author}%
	\addperiod\space%
	% 2) Title [Electronic Resource]
	\printfield[titlecase]{title}\setunit{\space}\printtext{[Electronic Resource]}%
	\newunit\newblock%
	% 3) Year
	\printfield{year}%
	\newunit\newblock%
	% 4) URL
	\printfield{url}\setunit{\space}%
	\usebibmacro{my:urldate}%
}


% --- Force AUTHOR-FIRST layout ---
\makeatletter
\AtBeginDocument{%
	\DeclareBibliographyDriver{article}{%
		\usebibmacro{bibindex}%
		\usebibmacro{begentry}%
		\usebibmacro{my:core}%  ← your logic
		\usebibmacro{finentry}%
	}%
	\DeclareBibliographyDriver{inproceedings}{%
		\usebibmacro{bibindex}%
		\usebibmacro{begentry}%
		\usebibmacro{my:core}%  ← your logic
		\usebibmacro{finentry}%
	}%
	\DeclareBibliographyDriver{incollection}{%
		\usebibmacro{bibindex}%
		\usebibmacro{begentry}%
		\usebibmacro{my:core}%  ← your logic
		\usebibmacro{finentry}%
	}%
	\DeclareBibliographyDriver{book}{%
		\usebibmacro{bibindex}%
		\usebibmacro{begentry}%
		\usebibmacro{my:core}%  ← your logic
		\usebibmacro{finentry}%
	}%
	\DeclareBibliographyDriver{online}{%
		\usebibmacro{bibindex}%
		\usebibmacro{begentry}%
		\usebibmacro{my:online}%  ← your logic
		\usebibmacro{finentry}%
	}%
}
\makeatother

