\chapter*{Аннотация}

В данной диссертационной работе рассматриваются задачи доменной адаптации (ДА) нейронных сетей для сегментации трёхмерных медицинских изображений. Работа структурирована на четыре основных части.

В первой части исследуется влияние доменного сдвига на различные слои сегментационной модели. Установлено, что карты признаков начальных слоёв особенно чувствительны к изменениям распределения данных, что приводит к существенному снижению точности. Показано, что адаптация именно этих слоёв даёт лучшие результаты. Также предлагается метод SpotTUnet, автоматически определяющий наиболее подверженные доменному сдвигу слои и обеспечивающий интерпретируемость процесса адаптации модели.

Вторая часть посвящена доменным сдвигам в изображениях компьютерной томографии (КТ), вызванным вариациями ядер реконструкции. Предлагаются два метода: FBPAug --- метод аугментации, моделирующий процесс реконструкции, и F-Consistency --- метод адаптации на основе данных, использующий попарные КТ-изображения, реконструированные с различными ядрами. Оба подхода значительно повышают согласованность предсказаний и обобщающую способность моделей.%, в том числе в задаче сегментации признаков COVID-19.

В третьей части представлены эталонный набор M3DA и коллекция данных BGP, предназначенные для оценки и повышения надёжности методов ДА. M3DA демонстрирует ограничения существующих методов и предоставляет комплексную платформу для разработки устойчивых решений при различных сценариях доменного сдвига. Также публикуется аннотированная коллекция данных BGP для задачи сегментации глиобластомы, предназначенная, в частности, для оценки алгоритмов адаптации в реалистичных клинических условиях с высокой неоднородностью данных.

Четвёртая часть посвящена задаче обнаружения аномалий в трёхмерных медицинских изображениях. Показано, что существующие методы обладают серьёзными ограничениями и часто выдают высокий уровень ложноположительных срабатываний. Для разработки более устойчивых моделей сегментации, способных распознавать нетипичные случаи в реальной клинической практике, нами создаётся и публикуется соответствующий эталонный набор данных.

%Наиболее гибкий и общий подход к моделированию случайных возмущений в условных задачах оптимизации — использование \emph{Вероятностных Ограничений} (ВО). Они позволяют наперед задавать вероятность нарушения исходных ограничений и избегать излишней консервативности. В большинстве случаев, ВО не выражаются через элементарные функции, что затрудняет их использование в численных методах. Чтобы обойти это, были предложены различные аппроксимации с использованием данных, включая Аппроксимацию Сценариями (АС). Несмотря на теоретические гарантии, необходимое количество данных (сценариев) велико, что усложняет оптимизацию. В данной работе предлагаются методы и алгоритмы для оценки значения ВО и решения задач оптимизации с ВО, требующие меньше данных для получения приближенного решения, допустимого для ВО с высокой вероятностью.

%Метод для оценки значения ВО  разработан с использованием оптимизационно-статистического подхода адаптивной сыборки по значимости и продемонстрирован на примере оценки допустимости текущего режима генерации в элетрических сетях. Предложены подходы к подбору сценариев для АС для линейного программирования в случае аддитивных и мультипликативных возмущений, выделяющий область избыточных сценариев, не приводящих к выходу за допустимую область. Эффективность подходов продемонстрирована на примере задачи оптимального распределения потоков электроэнергии.

%Результаты исследования показали значительное улучшение скорости сходимости дисперсии оценки к минимуму и снижение зависимости от размерности задачи до $O(\sqrt{\log K})$, где $K$ — количество детерменированных ограничений. Предыдущие результаты в области показывали линейную зависимость - $O(K)$. Численные эксперименты показали, что предложенный метод более устойчив в определенных синтетических постановках по сравнению с другими современными методами и более эффективен в приложениях из энергетических сетей. Для оптимизации с ВО удалось теоретически доказать уменьшение количества необходимых сценариев для получения допустимого решения с высокой вероятностью; численно, количество сценариев сократилось, в среднем, в 2 раза.