\chapter*{Аннотация}

В данной диссертационной работе рассматриваются задачи доменной адаптации нейронных сетей для сегментации трёхмерных медицинских изображений. Работа структурирована на четыре основных части.

В первой части исследуется влияние доменного сдвига на различные слои сегментационной модели. Установлено, что карты признаков начальных слоёв особенно чувствительны к изменениям распределения данных, что приводит к существенному снижению точности. Показано, что адаптация именно этих слоёв даёт лучшие результаты. Также предлагается метод SpotTUnet, автоматически определяющий наиболее подверженные доменному сдвигу слои и обеспечивающий интерпретируемость процесса адаптации модели.

Вторая часть посвящена доменным сдвигам в изображениях компьютерной томографии (КТ), вызванным вариациями ядер реконструкции. Предлагаются два метода: FBPAug --- метод аугментации, моделирующий процесс реконструкции, и F-Consistency --- метод адаптации на основе данных, использующий попарные КТ-изображения, реконструированные с различными ядрами. Оба подхода значительно повышают согласованность предсказаний и обобщающую способность моделей.%, в том числе в задаче сегментации признаков COVID-19.

В третьей части представлены эталонный набор M3DA и коллекция данных BGP, предназначенные для оценки и повышения надёжности методов доменной адаптации. M3DA демонстрирует ограничения существующих методов и предоставляет комплексную платформу для разработки устойчивых решений при различных сценариях доменного сдвига. Также публикуется аннотированная коллекция данных BGP для задачи сегментации глиобластомы, предназначенная, в частности, для оценки алгоритмов адаптации в реалистичных клинических условиях с высокой неоднородностью данных.

Четвёртая часть посвящена задаче обнаружения аномалий в трёхмерных медицинских изображениях. Показано, что существующие методы обладают серьёзными ограничениями и часто дают высокий уровень ложноположительных срабатываний. Для разработки более устойчивых моделей сегментации, способных распознавать нетипичные случаи в реальной клинической практике, нами создаётся и публикуется соответствующий эталонный набор данных.
