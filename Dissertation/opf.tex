\chapter{Optimal Power Flow - OPF}
\label{chap:opf}

\section{Introduction}

Optimal Power Flow (OPF), proposed by Carpentier in 1962, is an optimization problem that seeks to find the cheapest generation set points over the set of generating units to satisfy the demand in a power system and meet technical limits that are given by Power Flow Equations, described in Chapter \ref{sec:PFE}, for a single time snapshot which is called steady-state OPF \cite{carpentier1962contribution}. This optimization problem is fundamental for a power system. Numerous modifications can help to obtain a range of results from intra-day generation schedule to the power system expansion plan \cite{bai2008semidefinite, molzahn2014investigation}.

The OPF is non-convex, due to the non-linearity of PFEs \eqref{eq:power_injections} and, generally, NP-hard \cite{lavaei2011zero}. Typically, instances of such class of problems are solved with iterative methods that require initial guess and guarantee local optimality \cite{molzahn2018towards}. Methods based on Linear Programming, Quadratic Programming, Newton's method, decomposition methods and interior-point based methods are well applied for the OPF. % cite for each method

This chapter is devoted to the OPF problem: its objective, variables, constraints, application-driven linear relaxation -- Direct Current OPF (DC-OPF) and dynamic OPF that models several temporarily interconnected timestamps of the system.

\section{Optimization problem}

A general constrained optimization problem problem is formulated as follows {\cite{red} BT, introduction to optimization}:

\begin{equation}
    \begin{aligned}
        \min            & J(x, u)    \\
        \texttt{s. t.}~ & g(x, u) = 0 \\
                        & h(x, u) \leq 0
    \end{aligned}
    \label{eq:constrained_opt}
\end{equation}
The vector $x \in \mathbb{R}^{n_x}$ is the vector of dependent variables, $u \in \mathbb{R}^{n_u}$ is the vector of decision variables, the cost function is scalar: $J: \mathbb{R}^{n_x + n_u} \to \mathbb{R}$. The equality and inequality constraints are given as the vector functions $g: \mathbb{R}^{n_x + n_u} \to \mathbb{R}^{m}$, $h: \mathbb{R}^{n_x + n_u} \to \mathbb{R}^{l}$. 

In OPF, the vector of independent variables consists of active power generation, voltage phases and magnitudes: $u = (p_g, v_m, \theta) \in \mathbb{R}^{|\mathcal{B}_G| + |\mathcal{B}| + |\mathcal{B}|}$. The objective function $J(x, u)$ is given by the total generation cost for OPF: $J(x, u) = J(p_g) = \textit{cost}(p_g)$, where $\textit{cost}: \mathbb{R}^{|\mathcal{B}_G|} \to \mathbb{R}$ is separable for each generator: 
\begin{equation}
\textit{cost} = \sum_{i \in \mathcal{B}_G} \textit{cost}_i (p_g^i).
\label{eq:objective_opf}
\end{equation}

\section{Constraints}

Constraints in OPF serve the purpose of ensuring meeting the demand, technical limits and transmission rules, induced by physical laws, are met within modeling. These constraints, as it was mentioned, classify into equality and inequality constraints and include power generation limits, power balance at buses, voltage limits, line limits and the reference bus constraints.

\subsection{Equality constraints}

Equality constraints include nodal power flow balance and PFEs \eqref{eq:power_injections}. For a bus $i \in \mathcal{B}$ let us denote $P_i^G, Q_i^G$ generated active and reactive power (0 if there is no generator at bus $i$) and $P_i^D, Q_i^D$ active and reactive power demand (0 if there is no load at bus $i$). Moreover, for a line $(i, j) \in \mathcal{L}$ let $P_{ij}, Q_{ij}$ be active and reactive line flows.  After a slight reformulation, these constraints are as follows: %{\color{red} find a reference preferably}
\begin{equation}
    \begin{aligned}
        P_i^G - P_i^D &= \sum_{j \in \textit{Adj}(i)} P_{ij}, ~i \in \mathcal{B} \\
        Q_i^G - Q_i^D &= \sum_{j \in \textit{Adj}(i)} Q_{ij}, ~i \in \mathcal{B} \\
    \end{aligned}
    \label{eq:nodal_flow}
\end{equation}
\begin{equation}
    \begin{aligned}
        P_{ij} &= V_i V_j \left[ G_{ij} \cos \theta_{ij} + B_{ij} \sin \theta_{ij} \right] - G_{ij} V_i^2, ~ij \in \mathcal{L} \\
        Q_{ij} &= V_i V_j \left[ G_{ij} \sin \theta_{ij} - B_{ij} \cos \theta_{ij} \right] + B_{ij} V_i^2, ~ij \in \mathcal{L} \\
    \end{aligned}
    \label{eq:line_flow}
\end{equation}

Assuming that reference bus is placed at bus $1 \in \mathcal{B}$, the slack bus constraint is
\begin{equation}
    \theta_1 = 0.
    \label{eq:slack_bus}
\end{equation}
\subsection{Inequality constraints}

Inequality constraints are induced by technical limits of lines (line ratings), generation technical limits and voltage limits for buses.

\paragraph{Generation limits}

The generator limits exist for ensuring that a generator is not requested to produce more power that it can and more power than minimal value to stay active. Let $\mathcal{G} \succ \mathcal{B}$ be the subset of buses that contain at least one generator. If there is multiple generators, the power generated from each one is summed and the results is the generation at bug $g \in \mathcal{G}$. Then limits are given, for $g \in \mathcal{G}$:
\begin{equation}
    \begin{aligned}
        P_g^{min} \leq &P_g^G \leq P_g^{max} \\
        Q_g^{min} \leq &Q_g^G \leq Q_g^{max} \\
        \label{eq:gen_lims}
    \end{aligned}
\end{equation}

\paragraph{Bus voltage limits}

To ensure safe operation of power grid, voltage is restricted by engineering limits. It is imposed on both magnitude and phase angle for each bus $i \in \mathcal{B}$:
\begin{equation}
    \begin{aligned}
        V_i^{min} \leq &V_i \leq V_i^{max} \\
        \theta_i^{min} \leq &\theta_i \leq \theta_i^{max} \\
        \label{eq:vol_lims}
    \end{aligned}
\end{equation}

\paragraph{Line rating}

Line rating, or line flow limits, are due to the wiring materials, components and length. Overhead lines that are subject to weather conditions have line rating that depend on the current weather condition and season \cite{fernandez2016review}. For a line $ij \in \mathcal{L}$ line rating are formulated as follows:
\begin{equation}
    P_{ij}^2 + Q_{ij}^2 \leq (S^{max}_{ij})^2.
    \label{eq:line_rating}
\end{equation}

\section{DC OPF Approximation}

In high voltage ($380$ kV, \cite{crucitti2005locating}) power grids, that are decribing a power system on inter-city country level, a common practice is to use Direct-Current OPF (DC-OPF) approximation. It does not assume direct current instead of alternating one. In fact, it has the following set of assumptions, recalling that $Y_{ij} = \frac{1}{R_{ij} + i X_{ij}}$ for a line $ij \in \mathcal{L}$:
\begin{enumerate}
    \item Voltage magnitudes are given in per units and approximately equal to one: $V_i \approx 1 ~ \forall i \in \mathcal{B}$
    \item Voltage phases does not differ much between buses: $\theta{ij} \approx 0, ~\forall ij \in \mathcal{L}$
    \item Line resistance is negligible: $R_{ij} ~ \forall ij \in \mathcal{L}$
\end{enumerate}

This set of assumption allows one to rewrite the AC-OPF constraints. Firstly, let us note that elements of the admittance matrix under the assumption 3) will be as follows, recall that $Y = G = i B$:
\begin{equation}
    Y_{ij} = \frac{1}{R_{ij} + i X_{ij}} \approx \frac{1}{iX_{ij}} \approx B_{ij}.
    \label{eq:admittance matrix approx}
\end{equation}
Next, assumption 2) yield the following approximation for trigonometric functions over phase difference, recalling Taylor's formula,%{\color{red} matan reference}
\begin{equation}
    \begin{aligned}
        \sin \theta_{ij} &= \theta_{ij} + o(\theta_{ij}) \underset{\theta_{ij} \to 0}{\to} \theta_{ij}    \\
        \cos \theta_{ij} &= 1 - \theta^2_{ij} / 2 + o(\theta_{ij}^2) \underset{\theta_{ij} \to 0}{\to} 1    \\
    \end{aligned}
    \label{eq:phase diff approx}
\end{equation}

Finally, combining \eqref{eq:admittance matrix approx}, \eqref{eq:phase diff approx} with 1) ($V_i V_j \approx 1$) and considering power flow lines \eqref{eq:line_flow} one obtains:
\begin{equation}
    \begin{aligned}
        P_{ij} &\approx B_{ij} \sin \theta_{ij} , ~ij \in \mathcal{L} \\
        Q_{ij} &\approx B_{ij} , ~ij \in \mathcal{L}. \\
    \end{aligned}
    \label{eq:line_flow_keked}
\end{equation}

Note that power flows are not constant only for active power and governed by the phase difference $\theta_{ij}$. Since all nonlinearities were held in the line flows, the resulting optimization problem DC-OPF is linear, provided linear cost function.

Let us summarize the DC-OPF below:

\begin{equation}
    \begin{aligned}
        \min_{P^G, \theta}  & \textit{cost}(P^G) \\
        \textit{s.t. }      & P^G - P^D = B\theta
                            & P_i^{min} \leq P_i^G \leq P_i^{max}, ~ i \in \mathcal{B} \\ 
                            & \sum_{i \in \mathcal{B}} P^G_i = \sum_{i \in \mathcal{B}} P^D_i \\
                            & |\theta_{ij}| \leq \bar{\theta}_{ij}, ~ ij \in \mathcal{L}
    \end{aligned}
    \label{eq:dc-opf}
\end{equation}

\section{Dynamic DC-OPF and Automated Generation Control}

Dynamic DC-OPF setup consider several system snapshots bound with temporal ramp-up and ramp-down constraints on generation \cite{lou2019multi, machowski2020power}. Such models allows for intra-day modeling of a system, taking into account load schedule and dynamics of renewable energy generation which can be treated as fluctuations.

Power systems fluctuations typically occur both on the generation and demand side and arise from the intermittency of renewable energy generation, unstable demand from the customers, and intra-day electricity trading. Often the distribution of such fluctuations can be recovered based on historical data \cite{roald2017chance,owen2019importance}. 
Below we assume that the fluctuations are Gaussian with the mean and covariance recoverable from historical data \cite{anvari2016short,roald2017chance}. 
The fluctuations are involved the power balance equation and typically managed through primary and secondary control \cite{machowski2020power}.
Let us index a system time snapshot as $t = 0, 1, \dots, T$ and denote fluctuations from generation that lead to imbalance as $(\xi_i^G)^t$ and $(\xi_i^D)^t$ -- from demand.
A common strategy to mitigate power imbalance is adopting linear Automatic Generation Control (AGC). The generation and demand fluctuate as $(P_i^G)^t + (\xi_i^G)^t$ at bus $i=1, \dots, n_g$ and $P^D_i + (\xi_i^D)^t$ at node $i=1,\dots, n_d$ respectively. As we assume that the initial generation is balanced, i.e., $\sum_{i=1}^{n_g} (P^G_i)^0 = \sum_{i=1}^{n_d} P_i^D$, the purpose of AGC is to balance the aggregated uncertainty term $\xi^t = \sum_{i=1}^{n_g}(\xi^G_i)^t-\sum_{i=1}^{n_d}(\xi^D_i)^t,$ where $\xi^t$ is distributed as the sum of all the nodes' uncertainties. For example, if the uncertainties are independent and identically distributed (i.i.d) Gaussians with mean $\mu = 0$ and some variance $\sigma^2$, then $\xi^t$ follows a Gaussian distribution $\mathcal{N}\left(0, (n_g+n_d) \cdot \sigma^2 \right)$. Following \cite{roald2017chance,baros2021examining,mezghani2020stochastic},
the AGC recourse brings the generation to a new setpoint $(P^G)^{t+1} = (P^G)^t + \alpha \xi^t$. In the latter the participation factors $\alpha \in \mathbb{R}^{n_g}$ satisfy $~\alpha \geq 0, ~ \boldsymbol{1}^\top \alpha = 1$. It is easy to check that if the system is initially balanced, then the new generation set-point also ensures a balanced system. The power mismatch at timestamp $t=1, \dots, T$ is given by:
$$
    1^\top p^t - 1^\top p_d - \xi^t = 1^\top p^{t-1} - 1^\top p_d + \xi^t - \xi^t = 0.
    \label{eq:power-balance-agc}
$$
i.e., this control strategy keeps the system balanced. 

In order to take into account limited rate of change of genration production, ramp-up and ramp-down constraints are induced and limit change of generation from one step to the next one by $\Delta_i \geq 0, ~ i \in \mathcal{B}$. 

Summing up, the dynamic DC-OPF formalizes as follows for a temporal snapshots $t = 0, \dots, T$:

\begin{equation}
    \begin{aligned}
        \min_{P^G, \theta}  & \sum_{t=1}^T\textit{cost}((P^G)^t) \\
        \textit{s.t. }      & (P^G)^t - (P^D)^t = B\theta^t\\
                            & P_i^{min} \leq (P_i^G)^t \leq P_i^{max}, ~ i \in \mathcal{B}, ~ t \in 0, \dots, T \\ 
                            & \sum_{i \in \mathcal{B}} (P^G_i)^t = \sum_{i \in \mathcal{B}} (P^D_i)^t, ~ t \in 0, \dots, T \\
                            & |\theta_{ij}^t| \leq \bar{\theta}_{ij}, ~ ij \in \mathcal{L}, ~ t \in 0, \dots, T \\
                            & |(P^G_i)^t - (P^G_i)^{t-1}| \leq \Delta_i.\\
    \end{aligned}
    \label{eq:dyn-dc-opf}
\end{equation}
