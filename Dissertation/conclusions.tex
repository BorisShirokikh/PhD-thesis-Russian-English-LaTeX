\addcontentsline{toc}{chapter}{Conclusions}
\chapter*{Conclusion}


%1. Restate the Research Problem and Objectives
%Begin by briefly restating the research problem and the primary objectives of your thesis. This reminds readers of the central focus of your work.

% Example:
% "In this thesis, we addressed the challenge of integrating renewable energy sources into power systems, with a focus on improving reliability and optimizing power flow under uncertainty. Our primary objectives were to develop advanced methods for reliability assessment and to propose novel optimization techniques for chance-constrained optimal power flow."

In this thesis, we addressed the the challenges that occur in modeling power systems that experience high level of renewable energy penetration. Our main objective was to develop advanced statistical methods for current operating point reliability assessment and to propose novel optimization techniques for chance-constrained optimal power flow in various settings.

% 2. Summarize Key Findings
% Summarize the key findings and contributions of your research. Highlight the most important results and how they address the research problem.

% Example:
% "Our research led to the development of an adaptive importance sampling method, which significantly improves the accuracy and efficiency of risk estimation for reliability constraints. Additionally, the proposed A-priori Reduced Scenario Approximation (AR-SA) method reduces the number of samples required for reliable solutions in joint chance-constrained dynamic optimal power flow problems. These methods were validated through extensive simulations, demonstrating their effectiveness in handling uncertainties in power systems."

This research led to the development of adaptive importance sampling methods for grid reliability estimation and proposed new techniques for constructing scenario approximation for static and dynamic formulation of optimal power flow.

% 3. Discuss the Significance and Impact
% Discuss the broader significance and impact of your findings. Explain how your research contributes to the field and its potential real-world applications.

% Example:
% "The findings of this thesis have significant implications for the integration of renewable energy sources into power systems. By enhancing the reliability and efficiency of power system operations, our methods support the transition to sustainable energy solutions, contributing to global efforts to reduce greenhouse gas emissions and improve energy security. Furthermore, the proposed techniques can be applied to other areas of power systems engineering, offering a foundation for future advancements in the field."

The findings have significant implications on computing and estimating generation regimes of power grids, allowing for non-restrictive robust, statistical based calculation of generation regimes and efficient estimation of the latter's reliability. This contributes to safer transition to sustainable energy solutions, contributing to global efforts to reduce greenhouse gas emissions and improve energy security. Furthermore, the proposed techniques can be applied to other areas of power systems engineering and beyond, where uncertainty arises in optimization problems, offering a foundation for future advancements in the field.

% 4. Address Limitations
% Acknowledge any limitations of your research. Being transparent about the constraints and challenges you encountered adds credibility to your work.

% Example:
% "While our methods offer substantial improvements, there are limitations to consider. The adaptive importance sampling method relies on accurate physical information, which may not always be readily available. Additionally, the computational complexity of the AR-SA method, though reduced, may still pose challenges for extremely large-scale power systems."

Though the methods offer substantial improvements, there are limitations to consider. The method rely on high-voltage assumption, leading to a linear system/problems. However, the results can be generalized for non-linear cases by additional mathematical effort.

% 5. Suggest Future Work
% Suggest directions for future research based on your findings. Identify areas that require further investigation and how they can build on your work.

% Example:
% "Future research could explore the application of the adaptive importance sampling method to real-time power system operations, addressing the challenge of obtaining real-time physical data. Further development of the AR-SA method could focus on enhancing its scalability and applicability to even larger power grids. Additionally, integrating these methods with emerging technologies, such as smart grid systems and advanced forecasting techniques, presents a promising avenue for future work."

Future research could explore applications for non-linear settings and broader distribution class support. The former could be achieved by iterative constructions of convex restrictions and the latter with distributionally robust optimization techniques. 

% 6. Final Thoughts
% Conclude with a few final thoughts that encapsulate the essence of your research and its potential to inspire further advancements in the field.

% Example:
% "In conclusion, this thesis contributes to the ongoing efforts to integrate renewable energy sources into power systems more effectively. The developed methods not only address current challenges but also pave the way for future innovations. As the global energy landscape continues to evolve, the insights gained from this research will be instrumental in shaping a sustainable and resilient energy future."

In conclusion, this thesis contributes to the ongoing efforts to integrate renewable energy sources into power systems more effectively. The developed methods not only address current challenges but also pave the way for future innovations. As the global energy landscape continues to evolve, the insights gained from this research will be instrumental in shaping a sustainable and resilient energy future.



% \newpage
% % \addcontentsline{toc}{chapter}{List of figures}
% \listoffigures

% % \addcontentsline{toc}{chapter}{List of tables}
% \listoftables