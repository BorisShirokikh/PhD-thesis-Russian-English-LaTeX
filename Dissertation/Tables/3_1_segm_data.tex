

%\begin{table}[H]
%	\centering
%	\caption{Summary of the segmentation datasets. {\textit{Effective size} means the number of annotated images after appropriate filtering.} \label{tab:data_segm}}
%	\newcolumntype{C}{>{\centering\arraybackslash}X}
%	\resizebox{\textwidth}{!}{%
%	\begin{tabularx}{\textwidth}{CCCCCC}
%		\toprule
%		\multirow{1.5}{*}{\textbf{Dataset}} & \multirow{1.5}{*}{\textbf{Source }}& \makecell{\textbf{Effective} \\ \textbf{Size}} & \multirow{1.5}{*}{\textbf{Kernels}} & \multirow{1.5}{*}{\textbf{Annotations}} &\multirow{1.5}{*}{ \textbf{Split}} \\
%		\midrule
%		\multirow{4}{*}{COVID-train} & \makecell{Mosmed-1110 \\ \cite{morozov2020mosmeddata}} & 50 & \makecell{unknown \\ smooth} & COVID-19 mask & \multirow{4}{*}{\makecell{5-fold \\ cross-val}} \\
%		\cmidrule(lr){2-5}
%		& \makecell{MIDRC \\ \cite{tsai2021rsna}} & 112 & \makecell{B/L/BONE/ \\ STANDARD \\ (smooth)} & COVID-19 mask & \\
%		\midrule
%		COVID-test & Medseg-9 & 9 & \makecell{unknown \\ sharp} & \makecell{COVID-19 mask, \\ lungs mask} & \makecell{hold-out \\ test} \\
%		\bottomrule
%	\end{tabularx}}
%\end{table}


\begin{table}[h]
	\centering
	\caption{Summary of the segmentation datasets.\label{tab:data_segm}}
	
	%\resizebox{\textwidth}{!}{%
		\begin{tabular}{c c c c c c}
			\toprule
			Dataset & Source & Size & Kernels & Annotations & Split \\
			\midrule
			\multirow{2}{*}[-1.0em]{\textit{COVID-train}} & Mosmed-1110 & 50 & \makecell{unknown \\ smooth} & COVID-19 & \multirow{2}{*}[-1.5em]{\makecell{5-fold \\ cross-val}} \\
			\cmidrule(lr){2-5}
			& MIDRC & 112 & \makecell{B/L/BONE/ \\ STANDARD \\ (smooth)} & COVID-19 & \\
			\midrule
			\textit{COVID-test} & Medseg-9 & 9 & \makecell{unknown \\ sharp} & \makecell{COVID-19, \\ lungs} & \makecell{hold-out \\ test} \\
			\bottomrule
			
	\end{tabular}%}
\end{table}
