

\addcontentsline{toc}{chapter}{Introduction}
\chapter*{Introduction}
\label{chap:Introduction}


\paragraph{Background.}

The rapid advancements in Deep Learning (DL) have established a strong foundation for automating medical image segmentation tasks~\cite{lee2017deep}. Segmentation algorithms, particularly those based on Convolutional Neural Networks (CNNs), have achieved near human-level performance across a variety of clinical applications, including brain tumor, lung cancer, organ-at-risk, and liver tumor segmentation. These algorithms continue to evolve, offering increasingly accurate results.

In our previous work, we have demonstrated the practical impact of carefully developed DL methods in real clinical workflows. For instance, we contributed to the radiotherapeutic delineation process at the Moscow Gamma-Knife Center by developing a model that provides accurate brain metastases segmentation contours~\cite{shirokikh2022systematic}. Additionally, our work on automatic COVID-19 identification and severity triage has been successfully integrated into the Moscow healthcare system, streamlining patient management~\cite{goncharov2021ct,covid-program}.

Despite these successes, the widespread adoption of DL in medical imaging remains hindered by the significant drop in neural network performance when applied to data that differs from the training set. This challenge, known as \textbf{domain shift}, is particularly prevalent in medical imaging due to variations in scanner acquisition parameters, the introduction of new imaging modalities, and differences in patient populations. Our research has highlighted severe performance degradation due to domain shifts in tasks such as brain segmentation from Magnetic Resonance Imaging (MRI)~\cite{shirokikh2020first,zakazov2021anatomy}, COVID-19 segmentation from chest Computer Tomography (CT)~\cite{saparov2021zero,shimovolos2022adaptation} images, and several others~\cite{vasiliuk2023limitations,shirokikh2025m3da}.

Addressing these distributional differences between training and testing data is essential, requiring the development of robust Domain Adaptation (DA) methods. Although the importance of developing DA methods specifically in medical imaging has been emphasized in numerous studies~\cite{gulrajani2020search,uda_survey_2020,zhuang2020comprehensive,peng2018visda,zhang2021empirical}, we outline several critical gaps remaining in the literature.

One such gap is the lack of robustness to particular types of domain shifts, including variations in MRI intensity distributions, CT reconstruction kernels, radiation dosages, and presence/absence of the contrast agents. We systematically investigate these challenges and propose novel DA methods either tailored to specific domain shift scenarios or applied universally across all of them.

Additionally, we note the lack of standardized DA evaluation in 3D medical imaging. To this end, we propose the M3DA benchmark, a large-scale framework designed to comprehensively assess DA methods across multiple, clinically relevant domain shift scenarios, filling a crucial gap in the field. We also publish the Burdenko's Glioblastoma Progression (BGP) dataset, a systematic and annotated data collection for glioblastoma segmentation, enabling in-the-wild DA evaluation in highly heterogeneous clinical data.

To summarize, this thesis is dedicated to developing novel methods, constructing standardized benchmarks, and expanding publicly available datasets in the field of domain adaptation in 3D medical image segmentation, bringing DL models closer to robust, real-world clinical deployment.


\paragraph{Relevance of the work.}


The aforementioned DL advancements hold immense potential for improving diagnostic accuracy, reducing human error, and enhancing the overall efficiency of healthcare systems. However, the clinical application of DL models is significantly challenged by their susceptibility to domain shifts -- variations between the data on which models are trained and the data encountered in real-world clinical environments. Such shifts lead to a dramatic decline in model performance.

This work directly addresses the issue of domain shift in 3D medical image segmentation, contributing to the development of more robust and generalizable DL models. The methods proposed in this thesis, such as SpotTUnet, FBPAug, and F-Consistency, are designed to mitigate the adverse effects, ensuring that segmentation models maintain their accuracy on the target domains. Moreover, the introduction of new adaptation techniques, DA and Out-of-Distribution (OOD) benchmarks, and the annotated publicly available dataset for 3D medical imaging fills significant gaps in the current literature.



% \paragraph{Problem statement.}
%2. Problem Statement
%Identify the specific problem or gap in the current knowledge that your research addresses. This should be a concise statement that clearly defines the issue your thesis will tackle.
% Various algorithms have been developed to ensure grid reliability, utilizing methods ranging from machine learning to analytical and sampling-based approaches. Machine learning methods leverage historical data on weather, renewable generation, and grid operating parameters to estimate risks but are impractical for real-time operation due to their reliance on large datasets and extensive data collection times. Analytical approximation methods, which compute overload probabilities through integrals, often overestimate risks, particularly in rare events, compromising their practical efficiency. Sampling-based algorithms, such as Monte Carlo (MC) simulations, provide valuable alternatives for assessing reliability but struggle with the performance in evaluating rare, severe disturbances due to uniform exploration of fluctuation spaces.

% The Optimal Power Flow (OPF) problem, crucial for determining economically optimal power generation levels under given constraints, has several extensions to address uncertainty in power generation and consumption. Robust and chance-constrained formulations are popular, with the former assuming bounded uncertainty and the latter requiring high-probability satisfaction of security constraints. The Joint Chance-Constrained Optimal Power Flow (JCC-OPF) problem bounds the probability of security constraint failures but is computationally hard even under linear security limits and Gaussian uncertainty. Tractable convex approximations often yield conservative solutions unsuitable for practical operations. Scenario and Sample Average Approximations, which replace stochastic elements with deterministic inequalities, can handle non-Gaussian uncertainties but may require a large number of samples, complicating their application in large-scale grids.

% Additionally, the discrete-time dynamic chance-constrained OPF problem addresses optimal generation set-points over sequential timestamps, incorporating ramp-up and ramp-down constraints to manage the rate of power output changes. Automatic Generation Control (AGC) aids in efficient power dispatch, yet solving the chance-constrained problem for arbitrary distributions and joint technical limits remains computationally infeasible. Data-driven approximations, such as Scenario Approximation (SA) and Sample Average Approximation (SAA), although effective, are often computationally prohibitive when high accuracy is needed, necessitating extensive scenario reduction studies. This complexity highlights the need for improved methods to handle the uncertainties and operational challenges posed by the integration of renewable energy sources into power systems.

% Summing up, modern power system are influenced by various uncertainty sources and require modern data-driven and data-efficient methods to, firstly, estimate reliability of the current power system state, secondly, reliably control the conventional generators to reach the most economically efficient state, simultaneously satistying demand and meeting technical constraints.

%Example:
%"Despite significant advancements in power systems, challenges remain in improving the efficiency and stability of power grids, particularly with the integration of renewable energy sources. This research aims to address these challenges by exploring new circuit designs and control strategies."
\paragraph{Dissertation goals.}

The primary goal of this thesis is to \textbf{address the challenges posed by domain shifts in 3D medical image segmentation}, which are a significant obstacle to the reliable and accurate performance of DL models in real-world clinical settings. By investigating various sources of domain shifts, particularly in MRI and CT images, the thesis aims to develop robust methods that improve the generalization of segmentation models across different imaging domains. This work not only focuses on enhancing the performance of models through developing fine-tuning and domain adaptation (DA) methods but also introduces novel benchmarks and a dataset to ensure that these models can effectively handle out-of-distribution (OOD) data, which is critical for the safe deployment of deep learning (DL) in medical applications.

In accordance to this goal, the following tasks were set:


\begin{enumerate}
    \item To investigate the domain shift effects on segmentation model performance, particularly in MRI and CT images.
    \item Develop and evaluate new robust methods to mitigate domain shifts, i.e., to improve the consistency and generalization of deep neural network segmentation models across varied imaging conditions.
    \item Establish benchmarks and datasets for DA and OOD detection in medical imaging to facilitate the evaluation and development of robust methods for 3D medical image segmentation.
    \item Examine existing DA and OOD detection methods and propose or indicate improvements.
\end{enumerate}


% 5. Scope and Limitations
% Define the scope of your research, including what will and will not be covered. This helps to set clear boundaries and manage the expectations of your readers.

% Example:
% "The scope of this research is limited to the development and analysis of circuit designs and control strategies within the context of renewable energy integration. It does not cover other aspects of power systems such as economic analysis or policy implications."

% 6. Structure of the Thesis
% Provide an outline of the subsequent chapters and briefly describe their content. This helps readers understand the organization of your thesis and the logical flow of your research.

% Example:
% "The thesis is organized into six chapters. Chapter 2 reviews the relevant literature on power systems and renewable energy integration. Chapter 3 details the theoretical framework and methodologies used in the research. Chapter 4 presents the development of the proposed circuit designs. Chapter 5 discusses the control strategies for renewable energy integration. Chapter 6 evaluates the performance of the proposed solutions through simulations and experimental results. Finally, Chapter 7 concludes the thesis with a summary of findings and suggestions for future research."

% 7. Summary
% Conclude the introduction with a brief summary that reiterates the importance of your research and sets the stage for the detailed exploration in the following chapters.

% Example:
% "In summary, this thesis aims to address critical challenges in power systems, particularly in the integration of renewable energy sources. By developing innovative circuit designs and control strategies, this research seeks to enhance the efficiency and stability of power grids, contributing to the advancement of sustainable energy solutions. The following chapters will delve into the theoretical foundations, methodologies, and empirical findings that support these objectives."

% Additional Tips
% Be Clear and Concise: Avoid unnecessary jargon and complex sentences. Aim for clarity and brevity to ensure your introduction is accessible to a broad audience.
% Engage the Reader: Start with a compelling statement or fact to capture the reader's interest.
% Cite Relevant Literature: Support your statements with references to key studies and authoritative sources in the field.
% By following these guidelines, you can craft an effective introduction chapter that provides a strong foundation for your PhD thesis on power systems.


\paragraph{Scientific novelty.}

The scientific novelty is built up from the following results:

\begin{enumerate}
	\item A novel gradient-based method, SpotTUnet, was proposed and implemented to automatically identify and fine-tune the neural network layers most susceptible to domain shifts in supervised domain adaptation settings, improving segmentation performance under limited data availability.
	\item A knowledge-driven augmentation technique, Filtered Back-Projection Augmentation (FBPAug), was developed for the first time to simulate the effect of CT reconstruction kernel variability, significantly improving segmentation consistency.
	\item A data-driven unsupervised domain adaptation method, F-Consistency, was developed for the first time to address CT reconstruction kernel variability by enforcing feature map similarity between paired images. The method achieves state-of-the-art performance under kernel-induced domain shifts and establishes a foundation for future self-supervised learning approaches using FBPAug-generated pairs and the F-Consistency criterion.
	\item A large-scale benchmark, M3DA, was introduced for evaluating unsupervised domain adaptation methods in 3D medical image segmentation, revealing critical limitations of existing approaches – none of which close more than 61\% of the performance gap between source and target domains.
	\item A new publicly available dataset was collected and published for primary glioblastoma segmentation in radiotherapy planning. It allows for testing domain adaptation methods in the close to clinical scenarios.
	\item A benchmark for out-of-distribution detection in 3D medical imaging was constructed for the first time, containing multiple challenge scenarios and exposing critical limitations of existing OOD detection methods.
\end{enumerate}

\paragraph{Theoretical and practical significance.}

The developed methods, including SpotTUnet, FBPAug, and F-Consistency, demonstrate strong potential for improving the robustness of medical segmentation models across diverse imaging protocols and clinical environments. The introduced M3DA and OOD detection benchmarks provide systematic tools for evaluating domain adaptation and reliability of segmentation algorithms in real-world scenarios. Furthermore, the published Burdenko Glioblastoma Progression dataset offers a clinically relevant testbed for developing and validating domain adaptation methods in radiotherapy planning. Altogether, the results of this dissertation contribute to advancing the safe deployment of deep learning in clinical practice and help bridge the gap between algorithmic development and medical application.

The majority of methods and benchmarks have publicly available source code:

\begin{enumerate}
	% \fontsize{10pt}{11pt}\selectfont
	\item First-layers fine-tuning: \href{https://github.com/kechua/DART20}{https://github.com/kechua/DART20}
	\item Domain Shift Anatomy: \hfill \hfill \linebreak \href{https://github.com/neuro-ml/domain_shift_anatomy}{https://github.com/neuro-ml/domain\_shift\_anatomy}
	\item Filtered Back-Projection Augmentation: \hfill \hfill \linebreak \href{https://github.com/STNLd2/FBPAug}{https://github.com/STNLd2/FBPAug}
	\item M3DA benchmark: \href{https://github.com/BorisShirokikh/M3DA}{https://github.com/BorisShirokikh/M3DA}
	\item {Burdenko's Glioblastoma Progression Dataset: \hfill \hfill \linebreak \href{https://www.cancerimagingarchive.net/collection/burdenko-gbm-progression/}{cancerimagingarchive.net/collection/burdenko-gbm-progression/}}
	\item Benchmark for Out-of-distribution detection on 3D medical images: \hfill \linebreak \href{https://github.com/francisso/OOD-benchmark}{https://github.com/francisso/OOD-benchmark}
	\item Intensity Histogram Features algorithm: \hfill \hfill \linebreak \href{https://github.com/BorisShirokikh/MOOD\_submission\_Sample-level\_AIRI}{github.com/BorisShirokikh/MOOD\_submission\_Sample-level\_AIRI}
\end{enumerate}

In addition to being published and open-sourced, FBPAug has been integrated into multiple products at IRA Labs Ltd, including a COVID-19 segmentation algorithm registered as Intellectual Property\footnote{\href{https://www.fips.ru/registers-doc-view/fips\_servlet?DB=EVM\&DocNumber=2021612647}{https://www.fips.ru/registers-doc-view/fips\_servlet?DB=EVM\&DocNumber=2021612647}}, further demonstrating the practical significance of this work.


\paragraph{Research methodology.}

Methodology included methods of linear algebra, analysis, probability theory, mathematical statistics, numerical optimization methods, software development and models of computer tomography.

%\newpage
% TODO: ???


\paragraph{Propositions for defense.}

\begin{enumerate}
	
	\item A novel supervised domain adaptation method, SpotTUnet, was developed to automatically identify and fine-tune layers most affected by domain shift. The method improved segmentation performance in MRI across diverse clinical domains and provided interpretable insights into layer-wise shift susceptibility. Used as a visualization or guidance tool, SpotTUnet contributed to the development of the further proposed methods, such as F-Consistency and IHF.
	\item A knowledge-driven augmentation technique, FBPAug, was proposed to address domain shifts caused by different CT reconstruction kernels. The method significantly increased prediction consistency across paired reconstructions, improving Dice score from 0.46 to 0.76 in COVID-19 lung segmentation. Furthermore, FBPAug was used to develop most of the commercially used CT segmentation algorithms, significantly improving their robustness.
	\item A data-driven unsupervised adaptation method, F-Consistency, was introduced to leverage paired CT images reconstructed with different kernels. By enforcing feature map similarity, the method achieved state-of-the-art results in kernel-induced domain shift settings, further increasing Dice score to 0.80.
	\item The M3DA benchmark was constructed for the large-scale evaluation of unsupervised DA methods in 3D medical image segmentation. It comprises four publicly available datasets and eight clinically relevant domain shift setups, including shifts across modalities, acquisition protocols, and contrast settings.
	\item A new dataset, Burdenko’s Glioblastoma Progression (BGP), was collected and published to evaluate segmentation models in a realistic DA scenario. It includes multi-sequence MRI scans from 180 patients acquired across multiple clinical sites and scanner vendors, enabling testing under heterogeneous acquisition variability.
	\item A benchmark for OOD detection in 3D medical segmentation was proposed, containing several clinically relevant challenge cases. It revealed fundamental limitations of existing OOD methods and established practical evaluation criteria for real-world robustness assessment.
	\item An effective OOD detection method, Intensity Histogram Features (IHF), was developed. With the lowest computational cost, it achieved top-2 ranking in the Medical Out-of-Distribution Challenge (MOOD) 2022 and 2023, making it a strong baseline and analysis tool for domain shifts in 3D medical images.

    % The convexity of the optimization problem is shown, stochastic gradient's expression is derived and, finally, the iterative method's convergence is shown. The method's performance is demonstrated against other estimation algorithms on power systems examples.
    % the proof was provided. The numerical demonstration is carried out on power systems test cases and compared to classical SA construction algorithms.
    %The Gaussianity assumption is demostrated to be valid using Shapiro-Wilks statistical tests on real time series. 
    % The theorem on solution reliability and the number of samples required for reduced problems is stated, the proof was provided. The numerical demonstration is carried out on power systems test cases and compared to advanced scenario reduction methods and ambiguous chance constrained method.
 
\end{enumerate}


\paragraph{Validation of the research results, reliability.}

The statements and conclusions formulated in the dissertation have received qualified approbation at the following international and Russian scientific conferences:

\begin{enumerate}
	\item Domain Adaptation and Representation Transfer MICCAI 2020 Workshop. Lima, Peru, October 2020.
    \item Cross-Modality Domain Adaptation for Medical Image Segmentation (challenge-associated satellite event at MICCAI). Strasbourg, France, September 2021.
    \item Conference on Medical Image Computing and Computer Assisted Intervention (MICCAI). Strasbourg, France, September 2021.
    \item Information Technologies and Systems (ITaS) conference. November 2021.
    \item Medical Out-of-Distribution Analysis Challenge (challenge-associated satellite event at MICCAI). Singapore, September 2022.
    \item Information Technologies and Systems (ITaS) conference. Istra, Russia, September 2023.
\end{enumerate}

The reliability of the results is ensured by comprehensive experimental evaluation across multiple datasets, diverse experimental settings, and multiple methods of statistical analysis, with all validation procedures in accordance with the leading work in the area. The reproducibility of findings was validated through systematic benchmarking of multiple deep learning algorithms and repeated experiments to ensure statistical significance of the results. All the proposed methods are compared with existing methods for domain adaptation or OOD detection using measurable metrics.

The credibility is also confirmed by two publications of research results in Q1 peer-reviewed scientific journals, two publications in Q2 peer-reviewed journals, and three peer-reviewed conference proceedings, including one in proceedings of the Rank A conference. All of these journals and proceedings are WoS, Scopus indexed.

% \paragraph{Summary.}
% The integration of renewable energy sources into modern power systems presents both significant opportunities and challenges. While these sources are crucial for achieving carbon-free electricity generation and meeting global sustainability goals, their inherent variability and uncertainty pose risks to grid reliability and stability. This research identifies the limitations of current methods for managing these uncertainties, such as machine learning, analytical approximations, and sampling-based algorithms, particularly in the context of the Optimal Power Flow (OPF) problem and its extensions. In response, we propose novel methods, including adaptive importance sampling and A-priori Reduced Scenario Approximation (AR-SA), to enhance the efficiency and accuracy of reliability assessments and optimization in power systems under uncertainty. By addressing these challenges, the research aims to improve the operational reliability and economic efficiency of power systems, facilitating the successful integration of renewable energy sources and supporting the global transition towards a sustainable energy future.





% Template and formatting:
% All Skoltech theses have an abstract
% One thesis had chapter summary at the end of each chapter. Looks like a good idea
% Introduction is apparently just a regular chapter

\input{common/characteristic}

% \section*{Publications}

% \paragraph{Thesis publications.} The thesis is based on the following four Q1 publications:

% \begin{enumerate}
%     \item \fullcite{lukashevich2021importance}
%     \item \fullcite{lukashevich2021power}
%     \item \fullcite{lukashevich2023importance}
%     \item \fullcite{mitrovic2023data}
% \end{enumerate}

% In all the above papers, except the last one, the author was a principal contributor, who developed and implemented all listed algorithms, proved (in the third paper jointly with A. Bulkin) all supporting theorems and lemmas. In the last paper the applicant developed method for experimental section, took part in the developing comparison methodology.

%\newpage
% TODO: ???

\section*{Acknowledgments} 
The dissertation was completed at the {\thesisOrganizationEnNonTitle}.

I would like to thank everyone who has supported me throughout my academic journey and during the preparation of this thesis.

First and foremost, I am deeply thankful to my primary supervisor, Mikhail Belyaev, for his unwavering support and guidance during my Bachelor's, Master's, and Ph.D. studies. His insights into the research world and the knowledge he has shared have been invaluable in my academic and personal growth. Much of my scientific success is the result of our close collaboration and the work with his research team.

I am also deeply thankful to my collaborators and co-authors, who have greatly contributed to my scientific journey (in alphabetical order): Alexandra Dalechina, Daria Frolova, Mikhail Goncharov, Egor Krivov, Anvar Kurmukov, Ivan Oseledets, Maxim Panov, Mikhail Pautov, Maxim Pisov, Talgat Saparov, Alexey Shevtsov, Anton Vasiliuk, and Ivan Zakazov. I am fortunate to have had the opportunity to work with and learn from such brilliant people.

I am deeply thankful to my family for their constant help and belief in me.


