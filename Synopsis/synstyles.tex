%%% Изображения %%%
\graphicspath{{images/}{Synopsis/images/}}         % Пути к изображениям

%%% Макет страницы %%%
\geometry{a4paper, top=2cm, bottom=2cm, left=2.5cm, right=1cm, nofoot, nomarginpar} %, heightrounded, showframe
\setlength{\topskip}{0pt}   %размер дополнительного верхнего поля

%%% Интервалы %%%
%% Реализация средствами класса (на основе setspace) ближе к типографской классике.
%% И правит сразу и в таблицах (если со звёздочкой)
%\DoubleSpacing*     % Двойной интервал
%\OnehalfSpacing*    % Полуторный интервал
\SingleSpacing      % Одинарный интервал
%\setSpacing{1.42}   % Полуторный интервал, подобный Ворду (возможно, стоит включать вместе с предыдущей строкой)

%%% Выравнивание и переносы %%%
%% http://tex.stackexchange.com/questions/241343/what-is-the-meaning-of-fussy-sloppy-emergencystretch-tolerance-hbadness
%% http://www.latex-community.org/forum/viewtopic.php?p=70342#p70342
\tolerance 1414
\hbadness 1414
\emergencystretch 1.5em % В случае проблем регулировать в первую очередь
\hfuzz 0.3pt
\vfuzz \hfuzz
%\raggedbottom
%\sloppy                 % Избавляемся от переполнений
\clubpenalty=10000      % Запрещаем разрыв страницы после первой строки абзаца
\widowpenalty=10000     % Запрещаем разрыв страницы после последней строки абзаца
\brokenpenalty=4991     % Ограничение на разрыв страницы, если строка заканчивается переносом

% доп рекомендации от доктората: запрещаем переносы (авто и ручные) слов для лучшей читаемости + запрещаем разрывы абзацев картинками и табицами
\hyphenpenalty=10000
\exhyphenpenalty=10000
%\interlinepenalty=10000  % запрещает вставку плавающих объектов в абзац
\interlinepenalty=5000  % запрещает вставку плавающих объектов в абзац
\usepackage{float}
\floatplacement{figure}{H}
\floatplacement{table}{H}

%%% Колонтитулы %%%
\makeevenhead{plain}{}{}{}
\makeoddhead{plain}{}{}{}
\makeevenfoot{plain}{}{\thepage}{}
\makeoddfoot{plain}{}{\thepage}{}
\pagestyle{plain}

%%% Размеры заголовков %%%
\setsecheadstyle{\normalfont\large\bfseries}
\renewcommand*{\chaptitlefont}{\normalfont\large\bfseries}

%%% Подписи %%%
\setfloatadjustment{table}{%
    \setlength{\abovecaptionskip}{0pt}   % Отбивка над подписью
    \setlength{\belowcaptionskip}{0pt}   % Отбивка под подписью
}

%%% Отступы у плавающих блоков %%%
\setlength\textfloatsep{1ex}


%%% Блок управления параметрами для выравнивания заголовков в тексте %%%
\newlength{\otstuplen}
\setlength{\otstuplen}{\theotstup\parindent}
\ifnumequal{\value{headingalign}}{0}{% выравнивание заголовков в тексте
	\newcommand{\hdngalign}{\centering}                % по центру
	\newcommand{\hdngaligni}{}% по центру
	\setlength{\otstuplen}{0pt}
}{%
	\newcommand{\hdngalign}{}                 % по левому краю
	\newcommand{\hdngaligni}{\hspace{\otstuplen}}      % по левому краю
} % В обоих случаях вроде бы без переноса, как и надо (ГОСТ Р 7.0.11-2011, 5.3.5)

%%% Оформление заголовков глав, разделов, подразделов %%%
%% Работа должна быть выполнена ... размером шрифта 12-14 пунктов (ГОСТ Р 7.0.11-2011, 5.3.8). То есть не должно быть надписей шрифтом более 14. Так и поставим.
%% Эти установки будут давать одинаковый результат независимо от выбора базовым шрифтом 12 пт или 14 пт
\newcommand{\basegostsectionfont}{\fontsize{14pt}{16pt}\selectfont\bfseries}

\makechapterstyle{thesisgost}{%
	\chapterstyle{default}
	\setlength{\beforechapskip}{0pt}
	\setlength{\midchapskip}{0pt}
	\setlength{\afterchapskip}{\theintvl\curtextsize}
	\renewcommand*{\chapnamefont}{\basegostsectionfont}
	\renewcommand*{\chapnumfont}{\basegostsectionfont}
	\renewcommand*{\chaptitlefont}{\basegostsectionfont}
	\renewcommand*{\chapterheadstart}{}
	\ifnumgreater{\value{headingdelim}}{0}{%
		\renewcommand*{\afterchapternum}{.\space}   % добавляет точку с пробелом после номера раздела
	}{%
		\renewcommand*{\afterchapternum}{\quad}     % добавляет \quad после номера раздела
	}
	\renewcommand*{\printchapternum}{\hdngaligni\hdngalign\chapnumfont \thechapter}
	\renewcommand*{\printchaptername}{}
	\renewcommand*{\printchapternonum}{\hdngaligni\hdngalign}
}

\makeatletter
\makechapterstyle{thesisgostchapname}{%
	\chapterstyle{thesisgost}
	\renewcommand*{\printchapternum}{\chapnumfont \thechapter}
	\renewcommand*{\printchaptername}{\hdngaligni\hdngalign\chapnamefont \@chapapp} %
}
\makeatother

\chapterstyle{thesisgost}

\setsecheadstyle{\basegostsectionfont\hdngalign}
\setsecindent{\otstuplen}

\setsubsecheadstyle{\basegostsectionfont\hdngalign}
\setsubsecindent{\otstuplen}

\setsubsubsecheadstyle{\basegostsectionfont\hdngalign}
\setsubsubsecindent{\otstuplen}

\sethangfrom{\noindent #1} %все заголовки подразделов центрируются с учетом номера, как block

\ifnumequal{\value{chapstyle}}{1}{%
	\chapterstyle{thesisgostchapname}
	\renewcommand*{\cftchaptername}{\chaptername\space} % будет вписано слово Глава перед каждым номером раздела в оглавлении
}{}

%%% Интервалы между заголовками
\setbeforesecskip{\theintvl\curtextsize}% Заголовки отделяют от текста сверху и снизу тремя интервалами (ГОСТ Р 7.0.11-2011, 5.3.5).
\setaftersecskip{\theintvl\curtextsize}
\setbeforesubsecskip{\theintvl\curtextsize}
\setaftersubsecskip{\theintvl\curtextsize}
\setbeforesubsubsecskip{\theintvl\curtextsize}
\setaftersubsubsecskip{\theintvl\curtextsize}
