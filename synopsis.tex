% !TEX TS-program = xelatex
\RequirePackage[l2tabu,orthodox]{nag} % Раскомментировав, можно в логе получать рекомендации относительно правильного использования пакетов и предупреждения об устаревших и нерекомендуемых пакетах

% Откомментируйте, чтобы отключить генерацию закладок в pdf
% \PassOptionsToPackage{bookmarks=false}{hyperref}
%\documentclass[a5paper,10pt,twoside,openany,article]{memoir}
\documentclass[a4paper,14pt,oneside,openany]{memoir}
% не нужно делать зеркальные страницы, обычный А4 формат, там шрифт 14 т.к., при отправке на печать типография сама уменьшает масштаб


%%% Вывод типов ссылок в библиографии %%%
\makeatletter
\@ifundefined{c@mediadisplay}{
	\newcounter{mediadisplay}
	\setcounter{mediadisplay}{0}
	% 0 --- не делать ничего; надписи [Текст] и [Эл. ресурс] будут выводиться только в ссылках с заполненным полем `media`;
	% 1 --- автоматически добавлять надпись [Текст] к ссылкам с незаполненным полем `media`; таким образом, у всех источников будет указан тип, что соответствует требованиям ГОСТ
	% 2 --- автоматически удалять надписи [Текст], [Эл. Ресурс] и др.; не соответствует ГОСТ
	% 3 --- автоматически удалять надпись [Текст]; не соответствует ГОСТ
	% 4 --- автоматически удалять надпись [Эл. Ресурс]; не соответствует ГОСТ
}{}
\makeatother


% %%% INSTRUCTIONS %%%
% 1) Ссылки с архива, как и любого другого эл. ресурса, нужно будет определить как @online вместо дефолтного @article из школяра.
% 2) Обязательно нужно добавить два поля: url и urldate, чтобы появились требуемые ГОСТом адрес электронного ресурса и дата последнего посещения. Пример:
%		url={https://arxiv.org/abs/2502.17029},
% 		urldate={2025-09-21},
% 3) Поле типа "journal={arXiv preprint arXiv:1412.6980}" можно удалить, так как эта информация более не нужна. Опционально можно перенести содержимое в поле howpublished, но я так не делал.
% 4) Обязательно нужно было добавлять поле "media={eresource}", чтобы появилась требуемая ГОСТом пометка "[Electronic Resource]", но я добавил ниже код в \DeclareSourcemap{...}, который добавляет это поле в каждый @online автоматически.
          % общие настройки шаблона
\newif\ifsynopsis           % Условие, проверяющее, что документ --- автореферат


%%% Основные пакеты %%%
\usepackage{etoolbox}  % для проверки условий, оставляем для возможного расширения
\usepackage{comment}   % для возможности исключать блоки текста


%%% Поля и разметка страницы %%%
\usepackage{geometry}  % задание полей
\usepackage{pdflscape} % альбомные страницы с корректной ориентацией PDF
% \usepackage{lscape} % простая альбомная ориентация


%%% Математика %%%
\usepackage{amsmath,amssymb,amsfonts,amsthm,amscd}
\usepackage{mathtools}  % multlined и др.
\usepackage{xfrac}      % красивые дроби
\usepackage[
locale = DE,
list-separator       = {;\,},
list-final-separator = {;\,},
list-pair-separator  = {;\,},
list-units           = single,
range-units          = single,
range-phrase={\text{\ensuremath{-}}},
fraction-function    = \sfrac,
separate-uncertainty,
]{siunitx}[=v2]
\sisetup{inter-unit-product = \ensuremath{{}\cdot{}}}


%%%% Установки для размера шрифта 14 pt %%%%
%% Формирование переменных и констант для сравнения (один раз для всех подключаемых файлов)%%
%% должно располагаться до вызова пакета fontspec или polyglossia, потому что они сбивают его работу
\newlength{\curtextsize}
\newlength{\bigtextsize}
\setlength{\bigtextsize}{13.9pt}

\makeatletter
%\show\f@size    % неплохо для отслеживания, но вызывает стопорение процесса,
% если документ компилируется без команды  -interaction=nonstopmode
\setlength{\curtextsize}{\f@size pt}
\makeatother


%%% Кодировки и шрифты %%%
\usepackage{polyglossia}         % поддержка многоязычности
\setmainlanguage{english}        % основной язык
\setotherlanguage{russian}       % если вдруг нужен русский

\PassOptionsToPackage{no-math}{fontspec} % опция для fontspec, если нужны математические шрифты
\usepackage{fontspec} % шрифты для XeLaTeX

% Базовые шрифты (обычно нужно скачивать):
\setmainfont{Times New Roman} % ГОСТовский стандартный шрифт
\setsansfont{Arial}
\setmonofont{Courier New}[Scale=0.87] % подгоняет высоту под основной текст (по версии ChatGPT)
% Обеспечиваем кириллицу для этих семейств
\newfontfamily\cyrillicfont{Times New Roman}
\newfontfamily\cyrillicfontsf{Arial}
\newfontfamily\cyrillicfonttt{Courier New}[Scale=0.87]

% Публично доступные аналоги в Debian/Ubuntu:
%\setmainfont{Liberation Serif} % альтернативный свободный аналог Times
%\setsansfont{Liberation Sans}
%\setmonofont{Liberation Mono}[Scale=0.87]
%% Обеспечиваем кириллицу для этих семейств
%\newfontfamily\cyrillicfont{Liberation Serif}
%\newfontfamily\cyrillicfontsf{Liberation Sans}
%\newfontfamily\cyrillicfonttt{Liberation Mono}[Scale=0.87]


%%% Абзацы %%%
\indentafterchapter  % Красная строка после заголовков типа chapter
\usepackage{indentfirst}  % Отступ в первом абзаце после секций/глав
% TODO: как будто после секций не работает


%%% Цвета (если нужно для таблиц/графиков) %%%
\usepackage[dvipsnames]{xcolor}


%%% Таблицы %%%
\usepackage{longtable,ltcaption} % Длинные таблицы
\usepackage{multirow,makecell}   % Улучшенное форматирование таблиц
\usepackage{tabu, tabulary}      % таблицы с автоматически подбирающейся
% шириной столбцов (tabu обязательно
% до hyperref вызывать)

\makeatletter
%https://github.com/tabu-issues-for-future-maintainer/tabu/issues/26
\@ifpackagelater{longtable}{2020/02/07}{
	\def\tabuendlongtrial{%
		\LT@echunk  \global\setbox\LT@gbox \hbox{\unhbox\LT@gbox}\kern\wd\LT@gbox
		\LT@get@widths
	}%
}{}
\makeatother

\usepackage{threeparttable}      % автоматический подгон ширины подписи таблицы


%%% Общие утилиты %%%
%\usepackage{soulutf8}	% soulutf8.sty: warning: 29: This package is obsolete, use the soul package directly.
\usepackage{soul}		% Поддержка переносоустойчивых подчёркиваний и зачёркиваний
\usepackage{icomma}  	% Запятая в десятичных дробях
\usepackage[hyphenation,lastparline]{impnattypo} % Оптимизация расстановки переносов и длины последней строки абзаца


%%% Гиперссылки %%%
\let\CYRDZE\relax
%\usepackage[draft]{hyperref}
\usepackage{hyperref}
%\usepackage{hyperref}%[2012/11/06]


%%% Изображения %%%
\usepackage{graphicx}%[2014/04/25]  % Подключаем пакет работы с графикой
\usepackage{caption}                % Подписи рисунков и таблиц
\usepackage{subcaption}             % Подписи подрисунков и подтаблиц
\usepackage{pdfpages}               % Добавление внешних pdf файлов
\usepackage[export]{adjustbox}


%%% Счётчики %%%
\usepackage{aliascnt}
\usepackage[figure,table]{totalcount}   % Счётчик рисунков и таблиц
\usepackage{totcount}   % Пакет создания счётчиков на основе последнего номера
% подсчитываемого элемента (может требовать дважды
% компилировать документ)
% \usepackage{totpages}   % Счётчик страниц, совместимый с hyperref (ссылается
% на номер последней страницы). Желательно ставить
% последним пакетом в преамбуле


%%% Продвинутое управление групповыми ссылками (пока только формулами) %%%
\usepackage{cleveref}   % продвинутое управление ссылками
\usepackage{kvsetkeys}  % для корректной обработки пробелов в \label
% Добавление возможности использования пробелов в \labelcref
% https://tex.stackexchange.com/a/340502/104425
\makeatletter
\let\org@@cref\@cref
\renewcommand*{\@cref}[2]{%
	\edef\process@me{%
		\noexpand\org@@cref{#1}{\zap@space#2 \@empty}%
	}\process@me
}
\makeatother

\usepackage{placeins} % для \FloatBarrier


%%% User-specific packages %%%
\usepackage{upgreek} % прямые греческие ради русской традиции
\usepackage{pifont}         % adds nice "v" and "x" symbols
\usepackage{bbm}
%\usepackage[ruled]{algorithm2e}       % Пакеты общие для диссертации и автореферата
\synopsistrue                 % Этот документ --- автореферат
%\synopsisfalse                % Этот документ --- не автореферат

\input{Dissertation/dispackages} 		% Пакеты только для диссертации

%%% Вывод типов ссылок в библиографии %%%
\makeatletter
\@ifundefined{c@mediadisplay}{
	\newcounter{mediadisplay}
	\setcounter{mediadisplay}{0}
	% 0 --- не делать ничего; надписи [Текст] и [Эл. ресурс] будут выводиться только в ссылках с заполненным полем `media`;
	% 1 --- автоматически добавлять надпись [Текст] к ссылкам с незаполненным полем `media`; таким образом, у всех источников будет указан тип, что соответствует требованиям ГОСТ
	% 2 --- автоматически удалять надписи [Текст], [Эл. Ресурс] и др.; не соответствует ГОСТ
	% 3 --- автоматически удалять надпись [Текст]; не соответствует ГОСТ
	% 4 --- автоматически удалять надпись [Эл. Ресурс]; не соответствует ГОСТ
}{}
\makeatother


% %%% INSTRUCTIONS %%%
% 1) Ссылки с архива, как и любого другого эл. ресурса, нужно будет определить как @online вместо дефолтного @article из школяра.
% 2) Обязательно нужно добавить два поля: url и urldate, чтобы появились требуемые ГОСТом адрес электронного ресурса и дата последнего посещения. Пример:
%		url={https://arxiv.org/abs/2502.17029},
% 		urldate={2025-09-21},
% 3) Поле типа "journal={arXiv preprint arXiv:1412.6980}" можно удалить, так как эта информация более не нужна. Опционально можно перенести содержимое в поле howpublished, но я так не делал.
% 4) Обязательно нужно было добавлять поле "media={eresource}", чтобы появилась требуемая ГОСТом пометка "[Electronic Resource]", но я добавил ниже код в \DeclareSourcemap{...}, который добавляет это поле в каждый @online автоматически.
      		% Упрощённые настройки шаблона
%\input{Synopsis/synpackages}  % Пакеты для автореферата

%%% Вывод типов ссылок в библиографии %%%
\makeatletter
\@ifundefined{c@mediadisplay}{
	\newcounter{mediadisplay}
	\setcounter{mediadisplay}{0}
	% 0 --- не делать ничего; надписи [Текст] и [Эл. ресурс] будут выводиться только в ссылках с заполненным полем `media`;
	% 1 --- автоматически добавлять надпись [Текст] к ссылкам с незаполненным полем `media`; таким образом, у всех источников будет указан тип, что соответствует требованиям ГОСТ
	% 2 --- автоматически удалять надписи [Текст], [Эл. Ресурс] и др.; не соответствует ГОСТ
	% 3 --- автоматически удалять надпись [Текст]; не соответствует ГОСТ
	% 4 --- автоматически удалять надпись [Эл. Ресурс]; не соответствует ГОСТ
}{}
\makeatother


% %%% INSTRUCTIONS %%%
% 1) Ссылки с архива, как и любого другого эл. ресурса, нужно будет определить как @online вместо дефолтного @article из школяра.
% 2) Обязательно нужно добавить два поля: url и urldate, чтобы появились требуемые ГОСТом адрес электронного ресурса и дата последнего посещения. Пример:
%		url={https://arxiv.org/abs/2502.17029},
% 		urldate={2025-09-21},
% 3) Поле типа "journal={arXiv preprint arXiv:1412.6980}" можно удалить, так как эта информация более не нужна. Опционально можно перенести содержимое в поле howpublished, но я так не делал.
% 4) Обязательно нужно было добавлять поле "media={eresource}", чтобы появилась требуемая ГОСТом пометка "[Electronic Resource]", но я добавил ниже код в \DeclareSourcemap{...}, который добавляет это поле в каждый @online автоматически.
        % Упрощённые настройки шаблона

\input{common/newnames}       % Новые переменные, которые могут использоваться во всём проекте

\input{common/data}           % Основные сведения
%%% Шаблон %%%
% Абзацный отступ. Должен быть одинаковым по всему тексту и равен пяти знакам (ГОСТ Р 7.0.11-2011, 5.3.7).
\AtBeginDocument{\setlength{\parindent}{2.5em}}


%%% Таблицы %%%
\DeclareCaptionLabelSeparator{tabsep}{\tablabelsep} % нумерация таблиц
%\DeclareCaptionFormat{split}{\splitformatlabel#1\par\splitformattext#3}

\captionsetup[table]{
	format=\tabformat,                % формат подписи (plain|hang)
	font=normal,                      % нормальные размер, цвет, стиль шрифта
	skip=0pt,                         % отбивка под подписью
	parskip=0pt,                      % отбивка между параграфами подписи
	position=above,                   % положение подписи
	justification=\tabjust,           % центровка
	indent=\tabindent,                % смещение строк после первой
	labelsep=tabsep,                  % разделитель
	singlelinecheck=\tabsinglecenter, % не выравнивать по центру, если умещается в одну строку
}


%%% Рисунки %%%
\DeclareCaptionLabelSeparator{figsep}{\figlabelsep} % нумерация рисунков

\captionsetup[figure]{
	format=plain,                     % формат подписи (plain|hang)
	font=normal,                      % нормальные размер, цвет, стиль шрифта
	skip=.0pt,                        % отбивка под подписью %%% skip=6pt, чтобы подпись не прилипала к рисунку.
	parskip=.0pt,                     % отбивка между параграфами подписи
	position=below,                   % положение подписи
	singlelinecheck=true,             % выравнивание по центру, если умещается в одну строку
	justification=\tabjust,
	%justification=centerlast,		  % в ГОСТе нет требования - так что пофиг.
	labelsep=figsep,                  % разделитель
}


%%% Настройки гиперссылок %%%
\hypersetup{
	linktocpage=true,           % ссылки с номера страницы в оглавлении, списке таблиц и списке рисунков
	%    linktoc=all,                % both the section and page part are links
	%    pdfpagelabels=false,        % set PDF page labels (true|false)
	plainpages=false,           % Forces page anchors to be named by the Arabic form  of the page number, rather than the formatted form
	colorlinks,                 % ссылки отображаются раскрашенным текстом, а не раскрашенным прямоугольником, вокруг текста
	linkcolor={linkcolor},      % цвет ссылок типа ref, eqref и подобных
	citecolor={citecolor},      % цвет ссылок-цитат
	urlcolor={urlcolor},        % цвет гиперссылок
	%    hidelinks,                  % Hide links (removing color and border)
	%    pdftitle={\thesisTitle},    % Заголовок
	pdftitle={\thesisTitlePDF},    % Заголовок
	pdfauthor={\thesisAuthor},  % Автор
	pdfsubject={\thesisSpecialtyNumber\ \thesisSpecialtyTitle},      % Тема
	%    pdfcreator={Создатель},     % Создатель, Приложение
	%    pdfproducer={Производитель},% Производитель, Производитель PDF
	pdfkeywords={\keywords},    % Ключевые слова
	pdflang={ru},
}


%%% Списки %%%
% Используем короткое тире (endash) для ненумерованных списков 
\renewcommand{\labelitemi}{\normalfont -} % ГОСТ 2.105-95, пункт 4.1.7, требует дефиса
%\renewcommand{\labelitemi}{\normalfont\bfseries{--}} % а так лучше смотрится

%%% Списки по ГОСТ (адаптация для англ. текста) %%%
% ГОСТ 2.105–95, п. 4.1.7: перечисления строчными буквами.
% В англ. диссертации — используем латиницу вместо кириллицы.
% 1-й уровень — латинские буквы: a), b), c) ...
%\renewcommand{\theenumi}{\alph{enumi}}
%\renewcommand{\labelenumi}{\theenumi)}
% 2-й уровень — арабские цифры: 1), 2), 3) ...
%\renewcommand{\theenumii}{\arabic{enumii}}
%\renewcommand{\labelenumii}{\theenumii)}
% 3-й уровень — латинские буквы: a), b), c) ...
% (на случай глубокой вложенности, хотя обычно не используется)
%\renewcommand{\theenumiii}{\alph{enumiii}}
%\renewcommand{\labelenumiii}{\theenumiii)}

\setlist{nosep,%                                    % Единый стиль для всех списков (пакет enumitem), без дополнительных интервалов.
	labelindent=\parindent,leftmargin=*%            % Каждый пункт, подпункт и перечисление записывают с абзацного отступа (ГОСТ 2.105-95, 4.1.8)
}


%%% Для параграфа Dissertation structure %%%
%%http://www.linux.org.ru/forum/general/6993203#comment-6994589 (используется totcount)
\makeatletter
\def\formtotal#1#2#3#4#5{%
	\newcount\@c
	\@c\totvalue{#1}\relax
	\newcount\@last
	\newcount\@pnul
	\@last\@c\relax
	\divide\@last 10
	\@pnul\@last\relax
	\divide\@pnul 10
	\multiply\@pnul-10
	\advance\@pnul\@last
	\multiply\@last-10
	\advance\@last\@c
	#2%
	\ifnum\@pnul=1#5\else%
	\ifcase\@last#5\or#3\or#4\or#4\or#4\else#5\fi
	\fi
}
\makeatother

\newcommand{\formbytotal}[5]{\total{#1}~\formtotal{#1}{#2}{#3}{#4}{#5}}

           % Стили общие для диссертации и автореферата

% \geometry{a4paper, top=2cm, bottom=2cm, left=2.5cm, right=1cm, nofoot, nomarginpar}
%%% Изображения %%%
\graphicspath{{images/}{Synopsis/images/}}         % Пути к изображениям

%%% Макет страницы %%%
\geometry{a4paper, top=2cm, bottom=2cm, left=2.5cm, right=1cm, nofoot, nomarginpar} %, heightrounded, showframe
\setlength{\topskip}{0pt}   %размер дополнительного верхнего поля

%%% Интервалы %%%
%% Реализация средствами класса (на основе setspace) ближе к типографской классике.
%% И правит сразу и в таблицах (если со звёздочкой)
%\DoubleSpacing*     % Двойной интервал
%\OnehalfSpacing*    % Полуторный интервал
\SingleSpacing      % Одинарный интервал
%\setSpacing{1.42}   % Полуторный интервал, подобный Ворду (возможно, стоит включать вместе с предыдущей строкой)

%%% Выравнивание и переносы %%%
%% http://tex.stackexchange.com/questions/241343/what-is-the-meaning-of-fussy-sloppy-emergencystretch-tolerance-hbadness
%% http://www.latex-community.org/forum/viewtopic.php?p=70342#p70342
\tolerance 1414
\hbadness 1414
\emergencystretch 1.5em % В случае проблем регулировать в первую очередь
\hfuzz 0.3pt
\vfuzz \hfuzz
%\raggedbottom
%\sloppy                 % Избавляемся от переполнений
\clubpenalty=10000      % Запрещаем разрыв страницы после первой строки абзаца
\widowpenalty=10000     % Запрещаем разрыв страницы после последней строки абзаца
\brokenpenalty=4991     % Ограничение на разрыв страницы, если строка заканчивается переносом

% доп рекомендации от доктората: запрещаем переносы (авто и ручные) слов для лучшей читаемости + запрещаем разрывы абзацев картинками и табицами
\hyphenpenalty=10000
\exhyphenpenalty=10000
%\interlinepenalty=10000  % запрещает вставку плавающих объектов в абзац
\interlinepenalty=5000  % запрещает вставку плавающих объектов в абзац
\usepackage{float}
\floatplacement{figure}{H}
\floatplacement{table}{H}

%%% Колонтитулы %%%
\makeevenhead{plain}{}{}{}
\makeoddhead{plain}{}{}{}
\makeevenfoot{plain}{}{\thepage}{}
\makeoddfoot{plain}{}{\thepage}{}
\pagestyle{plain}

%%% Размеры заголовков %%%
\setsecheadstyle{\normalfont\large\bfseries}
\renewcommand*{\chaptitlefont}{\normalfont\large\bfseries}

%%% Подписи %%%
\setfloatadjustment{table}{%
    \setlength{\abovecaptionskip}{0pt}   % Отбивка над подписью
    \setlength{\belowcaptionskip}{0pt}   % Отбивка под подписью
}

%%% Отступы у плавающих блоков %%%
\setlength\textfloatsep{1ex}


%%% Блок управления параметрами для выравнивания заголовков в тексте %%%
\newlength{\otstuplen}
\setlength{\otstuplen}{\theotstup\parindent}
\ifnumequal{\value{headingalign}}{0}{% выравнивание заголовков в тексте
	\newcommand{\hdngalign}{\centering}                % по центру
	\newcommand{\hdngaligni}{}% по центру
	\setlength{\otstuplen}{0pt}
}{%
	\newcommand{\hdngalign}{}                 % по левому краю
	\newcommand{\hdngaligni}{\hspace{\otstuplen}}      % по левому краю
} % В обоих случаях вроде бы без переноса, как и надо (ГОСТ Р 7.0.11-2011, 5.3.5)

%%% Оформление заголовков глав, разделов, подразделов %%%
%% Работа должна быть выполнена ... размером шрифта 12-14 пунктов (ГОСТ Р 7.0.11-2011, 5.3.8). То есть не должно быть надписей шрифтом более 14. Так и поставим.
%% Эти установки будут давать одинаковый результат независимо от выбора базовым шрифтом 12 пт или 14 пт
\newcommand{\basegostsectionfont}{\fontsize{14pt}{16pt}\selectfont\bfseries}

\makechapterstyle{thesisgost}{%
	\chapterstyle{default}
	\setlength{\beforechapskip}{0pt}
	\setlength{\midchapskip}{0pt}
	\setlength{\afterchapskip}{\theintvl\curtextsize}
	\renewcommand*{\chapnamefont}{\basegostsectionfont}
	\renewcommand*{\chapnumfont}{\basegostsectionfont}
	\renewcommand*{\chaptitlefont}{\basegostsectionfont}
	\renewcommand*{\chapterheadstart}{}
	\ifnumgreater{\value{headingdelim}}{0}{%
		\renewcommand*{\afterchapternum}{.\space}   % добавляет точку с пробелом после номера раздела
	}{%
		\renewcommand*{\afterchapternum}{\quad}     % добавляет \quad после номера раздела
	}
	\renewcommand*{\printchapternum}{\hdngaligni\hdngalign\chapnumfont \thechapter}
	\renewcommand*{\printchaptername}{}
	\renewcommand*{\printchapternonum}{\hdngaligni\hdngalign}
}

\makeatletter
\makechapterstyle{thesisgostchapname}{%
	\chapterstyle{thesisgost}
	\renewcommand*{\printchapternum}{\chapnumfont \thechapter}
	\renewcommand*{\printchaptername}{\hdngaligni\hdngalign\chapnamefont \@chapapp} %
}
\makeatother

\chapterstyle{thesisgost}

\setsecheadstyle{\basegostsectionfont\hdngalign}
\setsecindent{\otstuplen}

\setsubsecheadstyle{\basegostsectionfont\hdngalign}
\setsubsecindent{\otstuplen}

\setsubsubsecheadstyle{\basegostsectionfont\hdngalign}
\setsubsubsecindent{\otstuplen}

\sethangfrom{\noindent #1} %все заголовки подразделов центрируются с учетом номера, как block

\ifnumequal{\value{chapstyle}}{1}{%
	\chapterstyle{thesisgostchapname}
	\renewcommand*{\cftchaptername}{\chaptername\space} % будет вписано слово Глава перед каждым номером раздела в оглавлении
}{}

%%% Интервалы между заголовками
\setbeforesecskip{\theintvl\curtextsize}% Заголовки отделяют от текста сверху и снизу тремя интервалами (ГОСТ Р 7.0.11-2011, 5.3.5).
\setaftersecskip{\theintvl\curtextsize}
\setbeforesubsecskip{\theintvl\curtextsize}
\setaftersubsecskip{\theintvl\curtextsize}
\setbeforesubsubsecskip{\theintvl\curtextsize}
\setaftersubsubsecskip{\theintvl\curtextsize}
    % Стили для автореферата
%\input{Dissertation/disstyles}  		% Стили для диссертации

\input{Synopsis/userstyles}   % Стили для специфических пользовательских задач

%\newcommand\blank[1][\textwidth]{\noindent\rule[-.2ex]{#1}{.4pt}}
%\input{Dissertation/userstyles} 		% Стили для специфических пользовательских задач
%%% Библиография через biblatex + biblatex-gost, движок biber %%%
\usepackage{csquotes} % рекомендуется для biblatex

\usepackage[
backend=biber,         % движок
bibencoding=utf8,      % кодировка bib-файла
sorting=none,          % порядок литературы как в bib-файле
style=gost-numeric,    % стиль ГОСТ
language=autobib,      % язык цитат подстраивается под babel/polyglossia
autolang=other,        % многоязычная библиография
clearlang=true,         % сброс поля language если совпадает с основным языком
defernumbers=true,     % номера в списке литературы после всех цитирований
sortcites=true,        % сортировка нескольких цитат в скобках
doi=false,             % отключаем DOI
isbn=false             % отключаем ISBN
]{biblatex}


\ExecuteBibliographyOptions{defernumbers=true,refsection=none}


%%% Подключение файлов bib %%%
%\addbibresource[label=bl-external]{biblio/external_rebiber.bib}
\addbibresource[label=bl-external]{biblio/external.bib}
\addbibresource[label=bl-author]{biblio/author.bib}
\addbibresource[label=bl-registered]{biblio/registered.bib}


\input{biblio/source_maps}


%%% Тире как разделитель в библиографии традиционной руской длины:
\renewcommand*{\newblockpunct}{\addperiod\addnbspace\textemdash\space\bibsentence}


%%% В списке литературы обозначение одной буквой диапазона страниц англоязычного источника %%%
\DefineBibliographyStrings{english}{%
	pages = {p\adddot} %заглавность буквы затем по месту определяется работой самого biblatex
}


%%% Исправление длины тире в диапазонах %%%
% \cyrdash --- тире «русской» длины, \textendash --- en-dash
\DefineBibliographyExtras{russian}{%
	\protected\def\bibrangedash{%
		\textendash\penalty\value{abbrvpenalty}}% almost unbreakable dash
	\protected\def\bibdaterangesep{\bibrangedash}%тире для дат
}
\DefineBibliographyExtras{english}{%
	\protected\def\bibrangedash{%
		\textendash\penalty\value{abbrvpenalty}}% almost unbreakable dash
	\protected\def\bibdaterangesep{\bibrangedash}%тире для дат
}

%Set higher penalty for breaking in number, dates and pages ranges
\setcounter{abbrvpenalty}{10000} % default is \hyphenpenalty which is 12

%Set higher penalty for breaking in names
\setcounter{highnamepenalty}{10000} % If you prefer the traditional BibTeX behavior (no linebreaks at highnamepenalty breakpoints), set it to ‘infinite’ (10 000 or higher).
\setcounter{lownamepenalty}{10000}


%%% Макросы автоматического подсчёта количества авторских публикаций.
% Печатают невидимую (пустую) библиографию, считая количество источников.
% http://tex.stackexchange.com/a/66851/79756
%
\makeatletter
\newtotcounter{citenum}
\defbibenvironment{counter}
{\setcounter{citenum}{0}\renewcommand{\blx@driver}[1]{}} % begin code: убирает весь выводимый текст
{} % end code
{\stepcounter{citenum}} % item code: cчитает "печатаемые в библиографию" источники

\newtotcounter{citeauthorvak}
\defbibenvironment{countauthorvak}
{\setcounter{citeauthorvak}{0}\renewcommand{\blx@driver}[1]{}}
{}
{\stepcounter{citeauthorvak}}

\newtotcounter{citeauthorscopus}
\defbibenvironment{countauthorscopus}
{\setcounter{citeauthorscopus}{0}\renewcommand{\blx@driver}[1]{}}
{}
{\stepcounter{citeauthorscopus}}

\newtotcounter{citeauthorwos}
\defbibenvironment{countauthorwos}
{\setcounter{citeauthorwos}{0}\renewcommand{\blx@driver}[1]{}}
{}
{\stepcounter{citeauthorwos}}

\newtotcounter{citeauthorother}
\defbibenvironment{countauthorother}
{\setcounter{citeauthorother}{0}\renewcommand{\blx@driver}[1]{}}
{}
{\stepcounter{citeauthorother}}

\newtotcounter{citeauthorconf}
\defbibenvironment{countauthorconf}
{\setcounter{citeauthorconf}{0}\renewcommand{\blx@driver}[1]{}}
{}
{\stepcounter{citeauthorconf}}

\newtotcounter{citeauthor}
\defbibenvironment{countauthor}
{\setcounter{citeauthor}{0}\renewcommand{\blx@driver}[1]{}}
{}
{\stepcounter{citeauthor}}

\newtotcounter{citeauthorvakscopuswos}
\defbibenvironment{countauthorvakscopuswos}
{\setcounter{citeauthorvakscopuswos}{0}\renewcommand{\blx@driver}[1]{}}
{}
{\stepcounter{citeauthorvakscopuswos}}

\newtotcounter{citeauthorscopuswos}
\defbibenvironment{countauthorscopuswos}
{\setcounter{citeauthorscopuswos}{0}\renewcommand{\blx@driver}[1]{}}
{}
{\stepcounter{citeauthorscopuswos}}

\newtotcounter{citeregistered}
\defbibenvironment{countregistered}
{\setcounter{citeregistered}{0}\renewcommand{\blx@driver}[1]{}}
{}
{\stepcounter{citeregistered}}

\newtotcounter{citeauthorpatent}
\defbibenvironment{countauthorpatent}
{\setcounter{citeauthorpatent}{0}\renewcommand{\blx@driver}[1]{}}
{}
{\stepcounter{citeauthorpatent}}

\newtotcounter{citeauthorprogram}
\defbibenvironment{countauthorprogram}
{\setcounter{citeauthorprogram}{0}\renewcommand{\blx@driver}[1]{}}
{}
{\stepcounter{citeauthorprogram}}

\newtotcounter{citeexternal}
\defbibenvironment{countexternal}
{\setcounter{citeexternal}{0}\renewcommand{\blx@driver}[1]{}}
{}
{\stepcounter{citeexternal}}
\makeatother

\defbibheading{nobibheading}{} % пустой заголовок, для подсчёта публикаций с помощью невидимой библиографии
\defbibheading{pubgroup}{\section*{#1}} % обычный стиль, заголовок-секция
\defbibheading{pubsubgroup}{\noindent\textbf{#1}} % для подразделов "по типу источника"


%%% Команды для вывода списка литературы %%%
\newcommand*{\insertbibliofull}{\printbibliography[keyword=bibliofull]}
\newcommand*{\insertbiblioauthor}{\printbibliography[heading=pubgroup, keyword=biblioauthor, title=\bibtitleauthorEn]}
\newcommand*{\insertbiblioexternal}{\printbibliography[heading=pubgroup, keyword=biblioexternal, title=\bibtitlefullEn]}
\newcommand*{\insertbiblioregistered}{\printbibliography[heading=none, keyword=biblioregistered, title={}]}
%\newcommand*{\insertbiblioauthorimportant}{\printbibliography[heading=pubgroup, section=2, filter=papersregistered, title=\bibtitleauthorimportant]}

% Вариант вывода печатных работ автора, с группировкой по типу источника.
% Порядок команд `\printbibliography` должен соответствовать порядку в файле common/characteristic.tex
%\newcommand*{\insertbiblioauthorgrouped}{
	%	\section*{\bibtitleauthor}
	%	\ifsynopsis
	%	\printbibliography[heading=pubsubgroup, section=0, keyword=biblioauthorvak,    title=\bibtitleauthorvak,resetnumbers=true] % Работы автора из списка ВАК (сброс нумерации)
	%	\else
	%	\printbibliography[heading=pubsubgroup, section=0, keyword=biblioauthorvak,    title=\bibtitleauthorvak,resetnumbers=false] % Работы автора из списка ВАК (сквозная нумерация)
	%	\fi
	%	\printbibliography[heading=pubsubgroup, section=0, keyword=biblioauthorwos,    title=\bibtitleauthorwos,resetnumbers=false]% Работы автора, индексируемые Web of Science
	%	\printbibliography[heading=pubsubgroup, section=0, keyword=biblioauthorscopus, title=\bibtitleauthorscopus,resetnumbers=false]% Работы автора, индексируемые Scopus
	%	\printbibliography[heading=pubsubgroup, section=0, keyword=biblioauthorpatent, title=\bibtitleauthorpatent,resetnumbers=false]% Патенты
	%	\printbibliography[heading=pubsubgroup, section=0, keyword=biblioauthorprogram,title=\bibtitleauthorprogram,resetnumbers=false]% Программы для ЭВМ
	%	\printbibliography[heading=pubsubgroup, section=0, keyword=biblioauthorconf,   title=\bibtitleauthorconf,resetnumbers=false]% Тезисы конференций
	%	\printbibliography[heading=pubsubgroup, section=0, keyword=biblioauthorother,  title=\bibtitleauthorother,resetnumbers=false]% Прочие работы автора
	%}




\DeclareFieldFormat{number}{Is.\ #1}
% biblatex-gost defines \DeclareFieldFormat{number}{No.\ #1} by default.
% The line above resets it to print the ~raw number only~ Issue instead Number (GOST feature).
\DeclareFieldFormat{pages}{P.\ #1}
% For some reason, it does not print "P." in the case of a single page.

\newbibmacro*{my:urldate}{\printtext{(visited on \thefield{urlmonth}/\thefield{urlday}/\thefield{urlyear})}}


\newbibmacro*{my:core}{%
	% 1) Authors
	\printnames{author}%
	\setunit{\addspace/\space}%
	% 2) Title
	\printfield[titlecase]{title}%
	\setunit{\addspace//\space}%
	% 3) Journal
	\iffieldundef{journaltitle}%
		{\printfield[emph]{booktitle}}%
		{\printfield[emph]{journaltitle}}%
	\newunit\newblock%
	% 4) Year
	\printfield{year}%
	\newunit\newblock%
	% 5) Volume / Issue
	\iffieldundef{volume}{}{\printtext{\printfield{volume}}}%
	\iffieldundef{number}{}{\setunit{\addcomma\space}\printtext{\printfield{number}}}%
	\newunit\newblock%
	% 6) Pages
	\iffieldundef{pages}{}{\printtext{\printfield{pages}}}%
}


\newbibmacro*{my:online}{%
	% 1) Authors
	\printnames{author}%
	\setunit{\addspace/\space}%
	% 2) Title [Electronic Resource]
	\printfield[titlecase]{title}\setunit{\space}\printtext{[Electronic Resource]}%
	\newunit\newblock%
	% 3) Year
	\printfield{year}%
	\newunit\newblock%
	% 4) URL
	\printfield{url}\setunit{\space}%
	\usebibmacro{my:urldate}%
}


\newbibmacro*{my:reg}{%
	% 1) Authors
	\printnames{author}%
	\setunit{\addspace/\space}%
	% 2) Title
	\printfield[titlecase]{title}%
	\setunit{\addspace//\space}%
	% 3) Heading
	\printfield[emph]{heading}%
	\newunit\newblock%
	% 4) Reg ID
	\printtext{No. \thefield{requestnumber}, \thefield{publicationmonth}/\thefield{publicationday}/\thefield{publicationyear} (\thefield{prioritycountry})}%
	%\newunit\newblock%
	% 4) URL
	%\printfield{url}\setunit{\space}%
	%\usebibmacro{my:urldate}%
}


% --- Force AUTHOR-FIRST layout ---
\makeatletter
\AtBeginDocument{%
	\DeclareBibliographyDriver{article}{%
		\usebibmacro{bibindex}%
		\usebibmacro{begentry}%
		\usebibmacro{my:core}%  ← your logic
		\usebibmacro{finentry}%
	}%
	\DeclareBibliographyDriver{inproceedings}{%
		\usebibmacro{bibindex}%
		\usebibmacro{begentry}%
		\usebibmacro{my:core}%  ← your logic
		\usebibmacro{finentry}%
	}%
	\DeclareBibliographyDriver{incollection}{%
		\usebibmacro{bibindex}%
		\usebibmacro{begentry}%
		\usebibmacro{my:core}%  ← your logic
		\usebibmacro{finentry}%
	}%
	\DeclareBibliographyDriver{book}{%
		\usebibmacro{bibindex}%
		\usebibmacro{begentry}%
		\usebibmacro{my:core}%  ← your logic
		\usebibmacro{finentry}%
	}%
	\DeclareBibliographyDriver{online}{%
		\usebibmacro{bibindex}%
		\usebibmacro{begentry}%
		\usebibmacro{my:online}%  ← your logic
		\usebibmacro{finentry}%
	}%
	\DeclareBibliographyDriver{patent}{%
		\usebibmacro{bibindex}%
		\usebibmacro{begentry}%
		\usebibmacro{my:reg}%  ← your logic
		\usebibmacro{finentry}%
	}%
}
\makeatother

   %  Реализация пакетом biblatex через движок biber


\begin{document}

\input{Synopsis/title}        % Титульный лист
%\mainmatter                   % В том числе начинает нумерацию страниц арабскими цифрами с единицы
\mainmatter*                  % Нумерация страниц не изменится, но начнётся с новой страницы
\input{Synopsis/title_en}
\mainmatter*                  % Нумерация страниц не изменится, но начнётся с новой страницы

\section*{General description of work} 


\subsection*{Background}


The rapid advancements in Deep Learning (DL) have established a strong foundation for automating medical image segmentation tasks~\cite{lee2017deep}. Segmentation algorithms, particularly those based on Convolutional Neural Networks (CNNs), have achieved near human-level performance across a variety of clinical applications, including brain tumor, lung cancer, organ-at-risk, and liver tumor segmentation. These algorithms continue to evolve, offering increasingly accurate results.

%In our previous work, we have demonstrated the practical impact of carefully developed DL methods in real clinical workflows. For instance, we contributed to the radiotherapeutic delineation process at the Moscow Gamma-Knife Center by developing a model that provides accurate brain metastases segmentation contours~\cite{shirokikh2022systematic}. Additionally, our work on automatic COVID-19 identification and severity triage has been successfully integrated into the Moscow healthcare system, streamlining patient management~\cite{goncharov2021ct,covid-program}.

Despite this progress, the widespread deployment of CNN-based segmentation models in clinical practice remains significantly constrained. A central limitation is the lack of model robustness under domain shift -- a phenomenon where systematic differences in data distribution arise due to variation in scanner vendors, acquisition protocols, reconstruction algorithms, patient populations, or healthcare institutions. When a model trained on one data domain is evaluated on a different domain, even without any semantic changes, its performance often degrades dramatically. In medical imaging, this effect has been documented across multiple modalities, including Magnetic Resonance Imaging (MRI) and Computed Tomography (CT), and across various tasks such as brain tissue, tumor, and COVID-19 lesion segmentation~\cite{shirokikh2020first,zakazov2021anatomy,saparov2021zero,shimovolos2022adaptation}.

To address this issue, domain adaptation (DA) methods have been actively explored. DA focuses on transferring knowledge from a labeled source domain to an unlabeled or differently-distributed target domain. Although numerous DA techniques have been proposed specifically in medical imaging~\cite{gulrajani2020search,uda_survey_2020,zhuang2020comprehensive,peng2018visda,zhang2021empirical}, most methods are validated under narrow experimental settings that do not fully reflect the diversity and complexity of real clinical scenarios. Specifically, the medical imaging datasets used are often either private, too small for robust training and evaluation, or limited to a single domain shift source or synthetic task. Moreover, methods addressing severe and frequently encountered in practice domain shifts, such as varying reconstruction kernels in CT, are mostly ignored and underdeveloped.

Therefore, the field of medical image segmentation urgently requires domain adaptation techniques that are effective under realistic domain shifts, scalable to large 3D datasets, and rigorously evaluated in clinically representative conditions. This dissertation addresses these needs through the development of novel DA methods, introduction of robust evaluation benchmarks, and publication of heterogeneous datasets, with the overarching goal of enabling safe, accurate, and generalizable deployment of deep learning models in clinical medical imaging workflows.


\subsection*{Relevance of the work}

The research presented in this dissertation addresses an impactful problem in the field of artificial intelligence and medical imaging: the lack of robustness and generalization of deep learning models under domain shift. Despite remarkable progress in developing CNNs for medical image segmentation, their deployment in real-world clinical environments remains limited due to the sharp degradation in performance when data characteristics change, such as those caused by differences in scanner hardware, acquisition protocols, or reconstruction algorithms. This issue is particularly acute in 3D medical imaging, where annotated data is scarce and domain variability is high.

This thesis focuses on the development and evaluation of domain adaptation methods for 3D segmentation models. The developed methods, including SpotTUnet, FBPAug, and F-Consistency, demonstrate strong potential for improving the robustness of medical segmentation models across diverse imaging protocols and clinical environments. The introduced M3DA and OOD detection benchmarks provide systematic tools for evaluating domain adaptation and reliability of segmentation algorithms in real-world scenarios. Furthermore, the published Burdenko Glioblastoma Progression dataset offers a clinically relevant testbed for developing and validating domain adaptation methods in radiotherapy planning. Altogether, the results of this dissertation contribute to advancing the safe deployment of deep learning in clinical practice and help bridge the gap between algorithmic development and medical application. This research contributes to the methodological foundations of deep learning in healthcare by enhancing the adaptability and reliability of medical image segmentation models.


\subsection*{Dissertation goals}

This research is devoted to the development of robust DL methods for 3D medical image segmentation under domain shift. The work focuses on both algorithmic contributions to supervised and unsupervised DA, and on establishing standardized benchmarks and datasets to support realistic evaluation of generalization in medical imaging. The results are demonstrated on clinically relevant segmentation tasks using MRI and CT data.
To achieve this goal, the following tasks were set up and performed:

\begin{enumerate}
    \item Domain Shift Analysis in MRI and CT: Estimate the sensitivity of segmentation models to different sources of domain shift in MRI and CT images. Identify CNN layers most affected by domain shift and propose a strategies for automatic layer selection and interpretable domain adaptation.
    \item Development of DA Algorithms: Design a physics-informed augmentation method to simulate CT kernel-induced shift. Develop a data-driven unsupervised adaptation method based on enforcing feature map similarity between paired CT images. Implement and evaluate supervised and unsupervised adaptation pipelines in low-data and zero-shot settings.
    \item Benchmarking Robustness in 3D Medical Imaging at Scale: Construct a realistic and large-scale benchmark to systematically assess DA performance across varied domain shift scenarios in MRI and CT segmentation tasks. Collect and release a clinically realistic dataset -- a large-scale, annotated collection from multiple clinical centers and scanner vendors. Assess generalization of segmentation models under heterogeneous real-world conditions and evaluate state-of-the-art DA methods, reporting their limitations. Develop an OOD benchmark with several clinically relevant test cases and evaluate standard methods. % glioblastoma MRI collection
\end{enumerate}


% TODO: do we need newpage here?
%\newpage


\subsection*{Propositions for defense}

\begin{enumerate}
	
	\item A supervised DA method, SpotTUnet, is proposed for layer-wise adaptation in convolutional neural networks. The method automatically identifies and fine-tunes the layers most affected by domain shift using a proposed gradient-based approach for learning a layer selection policy. SpotTUnet also provides an interpretable visualization of domain shift sensitivity across network layers due to the proposed regularization criterion, contributing to theoretical understanding of neural network model generalization under domain shift.
	
	% \item A knowledge-driven augmentation technique, Filtered Back-Projection Augmentation (FBPAug), is developed to simulate domain shifts induced by variation in CT reconstruction kernels. The method is theoretically grounded in the mathematical model of CT image formation and mimics kernel-induced transformations in the sinogram space, resulting in controlled variation of image noise, resolution, and texture. Furthermore, an unsupervised DA method, F-Consistency, is proposed to exploit unlabeled paired CT images reconstructed with different kernels. It enforces similarity of intermediate feature representations by minimizing the mean squared error between corresponding feature maps. Empirical evaluation confirms that the proposed methods significantly improve segmentation accuracy, including a Dice score increase in COVID-19 lung segmentation and other clinically relevant tasks.
	
	\item A knowledge-driven augmentation technique, Filtered Back-Projection Augmentation (FBPAug), is developed to simulate domain shift induced by variation in CT reconstruction kernels. The method is theoretically grounded in the mathematical model of CT image acquisition and simulates kernel-induced transformations in the sinogram space, enabling controlled variation of image noise, resolution, and texture. Furthermore, an unsupervised DA method, F-Consistency, is proposed to leverage unlabeled paired CT images reconstructed with different kernels. The method enforces similarity of internal feature representations by minimizing the mean squared error between feature maps of corresponding images. Empirical evaluation confirms that the proposed methods significantly improve segmentation accuracy, including a Dice score increase in COVID-19 segmentation and other clinically relevant tasks.
    
    \item A large-scale benchmark, M3DA, is developed for the systematic evaluation of unsupervised DA methods in 3D medical image segmentation. A clinically realistic dataset, Burdenko’s Glioblastoma Progression (BGP), is collected and published to enable in-the-wild evaluation of segmentation models under heterogeneous MRI acquisition conditions. Additionally, a benchmark for OOD detection in 3D medical image segmentation is proposed, containing several clinically relevant challenge scenarios. These benchmarks provide a rigorous empirical basis for studying generalization and robustness of segmentation algorithms under real-world distributional variability.
 
\end{enumerate}


\subsection*{Scientific novelty}

% It seems that statements to defend can very much overlap with the sci novelty, at least in some examples they do.
The scientific novelty is built up from the following results:

\begin{enumerate}
    
    \item A novel gradient-based method, SpotTUnet, is proposed and implemented to automatically identify and fine-tune the neural network layers most susceptible to domain shifts in supervised DA settings, improving segmentation performance under limited data availability.
    
    \item A knowledge-driven augmentation technique, FBPAug, is developed for the first time to simulate the effect of CT reconstruction kernel variability, significantly improving segmentation consistency.
    
    \item A data-driven unsupervised domain adaptation method, F-Consistency, is developed for the first time to address CT reconstruction kernel variability by enforcing feature map similarity between paired images. The method achieves state-of-the-art performance under kernel-induced domain shifts and establishes a foundation for future self-supervised learning approaches using FBPAug-generated pairs and the F-Consistency criterion.
    
    \item A large-scale benchmark, M3DA, was introduced for evaluating unsupervised domain adaptation methods in 3D medical image segmentation, revealing critical limitations of existing approaches -- none of which close more than 61\% of the performance gap between source and target domains.
    
    \item A new publicly available dataset for primary glioblastoma segmentation in radiotherapy planning is collected and published. It allows for testing domain adaptation methods in the close to clinical scenarios.
    
    \item A benchmark for out-of-distribution detection in 3D medical imaging is constructed for the first time, containing multiple challenge scenarios and exposing critical limitations of existing OOD detection methods.
    
\end{enumerate}


\subsection*{Theoretical and practical significance}

From a theoretical perspective, the dissertation contributes novel formulations and learning principles for domain adaptation in 3D medical image segmentation. The proposed SpotTUnet method introduces a gradient-based criterion for identifying domain-shift-sensitive layers in CNNs, offering new insights into the structural behavior of deep models under distributional shift. The FBPAug method is grounded in the mathematics of computed tomography, providing a parametrically controlled augmentation technique to simulate domain shifts caused by reconstruction kernels. F-Consistency formulates an unsupervised learning objective based on the similarity of internal representations, establishing a principled approach for leveraging paired but unlabeled data. These developments enrich the foundations of machine learning, domain adaptation in particular, and support further study into robustness and model generalization.

On the practical side, the proposed methods have been shown to significantly improve segmentation robustness under clinically relevant domain shifts, including those caused by acquisition parameters, reconstruction algorithms, and institutional variability. The developed M3DA benchmark provides a comprehensive framework for empirical evaluation of DA algorithms, while the published BGP dataset enables realistic testing in radiotherapy planning scenarios. Additionally, the OOD detection benchmark and the IHF method contribute to building reliable diagnostic pipelines capable of identifying OOD inputs in practice. %Collectively, the results of this research support safer and more generalizable deployment of deep learning models in real-world medical imaging workflows, bridging the gap between theoretical advancements and clinical applicability.

%Classification of quantum phases by quantum means is potentially useful for condensed matter physics. The results on the convergence of VQE are important for the design of quantum optimization algorithms. The proposed method for validating the GHZ state is potentially useful for evaluating the properties of quantum devices in a simple manner.
%Practical significance of the work is supported by the usage of the results in delivering the RFBR grant No.\ 19-31-90159 ``Aspiranty''.


\subsection*{Research methodology}

The research employs machine learning and deep learning techniques to improve the generalization and robustness of CNNs in 3D medical image segmentation, focusing on both supervised and unsupervised learning paradigms. Methodologically, the work integrates tools from linear algebra, probability theory, mathematical statistics, and numerical optimization to develop and analyze domain adaptation algorithms.

Experimental evaluation relies on large-scale empirical studies using publicly available and newly introduced benchmarks and datasets. All methods are implemented in the Python programming language using established DL frameworks. Software components were designed for reproducibility and extensibility, supporting broader use in medical image analysis. Clinical relevance and task formulation were refined through collaboration with medical experts in radiology and radiation oncology.


\subsection*{The reliability}

The reliability of the results is ensured by comprehensive experimental evaluation across multiple datasets, diverse experimental settings, and multiple methods of statistical analysis, with all validation procedures in accordance with the leading work in the area. The reproducibility of findings was validated through systematic benchmarking of multiple deep learning algorithms and repeated experiments to ensure statistical significance of the results. Finally, all the proposed methods are compared with existing state-of-the-art using a broad set of metrics.

The statements and conclusions formulated in the dissertation have received qualified approbation at international and Russian conferences. The credibility is also confirmed by two publications of research results in Q1 peer-reviewed scientific journals, two publications in Q2 peer-reviewed journals, and three peer-reviewed conference proceedings, including one in proceedings of the Rank A conference -- all being WoS, Scopus indexed.
%Предложенные наборыданных были опубликованы.


\subsection*{Validation of the research results}

The main results of the work have been reported in the following scientific conferences and workshops:

\begin{enumerate}
    \item Domain Adaptation and Representation Transfer MICCAI 2020 Workshop. Lima, Peru, October 2020.
    \item Cross-Modality Domain Adaptation for Medical Image Segmentation (challenge-associated satellite event at MICCAI). Strasbourg, France, September 2021.
    \item Conference on Medical Image Computing and Computer Assisted Intervention (MICCAI). Strasbourg, France, September 2021.
    \item Information Technologies and Systems (ITaS) conference. Moscow, Russia, November 2021.
    \item Medical Out-of-Distribution Analysis Challenge (challenge-associated satellite event at MICCAI). Singapore, September 2022.
    \item Information Technologies and Systems (ITaS) conference. Istra, Russia, September 2023.
\end{enumerate}


\subsection*{Personal contribution}

All the results of the dissertation were obtained personally by the applicant or with his direct involvement. In particular, the applicant’s contribution includes setting research objectives, studying literature corresponding to the topic, designing and conducting extensive experiments, and analyzing the received results. The applicant, together with Dr. Mikhail Belyaev and the scientific supervisor, Prof. Ivan Oseledets, participated in the formulation of the dissertation aims and objectives. Discussion of the results and preparation for publication of these results were carried out jointly with co-authors.


\subsection*{Structure and volume of the dissertation} The dissertation consists of introduction, four chapters, conclusions, bibliography, list of symbols and abbreviations, list of tables, list of figures, list of 165 references. The full 
volume of the dissertation is 126 pages, including 26 figures and 18 tables.
% \input{common/characteristic}

\ifnumequal{\value{contnumfig}}{1}{}{\counterwithout{figure}{chapter}}
\ifnumequal{\value{contnumtab}}{1}{}{\counterwithout{table}{chapter}}



\section*{Contents of the dissertation}


\textbf{Introduction} substantiates the relevance of the research, which addresses the robustness and generalization of deep learning models in 3D medical image segmentation under domain shift. It formulates the main goals and objectives of the dissertation, outlines the applied research methodology, highlights the scientific novelty of the proposed domain adaptation approaches and evaluation frameworks, and emphasizes the theoretical and practical significance of the results. The section concludes by stating the key contributions and propositions submitted for defense, approbation and reliability of the work.


% %%%%%%%%%%%%%%% CHAPTER 1 %%%%%%%%%%%%%%%


The first chapter, \textbf{Domain Shift Anatomy}, focuses on understanding how domain shifts impact CNNs for medical image segmentation, particularly in MRI. This chapter studies the vulnerability of CNN layers to domain shift and proposes an adaptive fine-tuning approach to mitigate its negative effects.

We begin by evaluating a supervised DA scenario, where the training and target domains represent the same content (e.g., brain MRI), but differ due to acquisition settings and scanner vendors. For this study, we employ the CC359 dataset~\cite{cc359}, and the segmentation task is performed using a U-Net model~\cite{unet}. An initial set of experiments demonstrates that fine-tuning only the first layers of a CNN (closest to the input) leads to better adaptation results compared to full-network or last-layer tuning. This suggests that the low-level feature maps, which capture basic visual patterns like edges and textures, are the most sensitive to acquisition variability.
%We trained the model on the source domain and adapted to the target domain in a few-shot scenario, using 1 to 4 annotated target image slices.

\begin{figure}[!h]
	\centering
	\includegraphics[width=\textwidth]{Dissertation/Figures/2_mri/spottune_seg.pdf}
	\caption{SpotTUnet architecture. The U-Net backbone is pretrained on the source domain. This network is frozen (blue blocks) and has a copy (orange blocks) that is fine-tuned on the target domain. The policy network is simultaneously trained on the target domain to output binary decisions for each pair of blocks from the segmentation networks.}
	\label{fig:spottune_seg}
\end{figure}

We then introduce SpotTUnet (Figure~\ref{fig:spottune_seg}), a method that automatically identifies which layers to fine-tune based on a learnable layer selection policy. Formally, the policy mechanism could be stated as follows. We compute the output for the $l$-th network layer as


\begin{equation}
	x_l = I_l ( x ) F_l ( x_{l-1} ) + (1 - I_l ( x )) \tilde{F}_l ( x_{l-1} ),
\end{equation}

\noindent
where $F_l$ and $\tilde{F}_l$ represent the frozen and fine-tuned versions of the $l$-th block, and $I_l(x)$ is the binary indicator derived from the Policy Network’s prediction.

Policy Network is trained concurrently with the fine-tuning of the segmentation network. During each iteration, Policy Network makes ``soft'' predictions for each block, which we binarize (into $I_l(x)$) to determine whether to use the frozen or fine-tuned block. To ensure gradient flow through the binary decision-making process, we employ the Gumbel-Softmax trick~\cite{guo2019spottune}, enabling end-to-end gradient-based training.
% TODO: expand on gumbel softmax; cite Gumbel-Softmax

In addition to the standard segmentation loss ($\mathcal{L}_{segm}$), we introduce a regularization term to control the number of blocks that are fine-tuned:

\begin{equation}
	\mathcal{L} = \mathcal{L}_{segm} + \lambda \sum_{l=1}^N \left( 1 - I_l (x) \right).
\end{equation}

The $\mathbb{L}_1$ regularization term allows minimizing the total number of fine-tuned blocks, leading to a sparser, more deterministic and interpretable policy. As a result, SpotTUnet not only significantly outperforms other layer selection heuristics but also enables interpretable analysis of domain shift effects inside the model. Figure~\ref{fig:layerswise_template} illustrates a layer-wise shift sensitivity profile, highlighting the layers most affected by the domain differences.

\begin{figure}[h!]
	\centering
	\includegraphics[width=\textwidth]{Dissertation/Figures/2_mri/layerwise_template.pdf}
	\caption{Visualization of SpotTUnet's learned policy for different amounts of available target slices: 12 (upper-left), 45 (bottom-left), 270 (upper-right), and 800 (bottom-right).}
	\label{fig:layerswise_template}
\end{figure}

The results of this chapter lay the conceptual foundation for later methods introduced in the thesis: F-Consistency, which leverages SpotTUnet’s layer selection for adaptation in CT images, and IHF, which builds on the finding of early-layer sensitivity by extracting domain-specific features directly from the input image.


% %%%%%%%%%%%%%%% CHAPTER 2 %%%%%%%%%%%%%%%


The second chapter, \textbf{Mitigating Domain Shift from CT Reconstruction Kernels}, presents a study of a clinically significant source of domain shift in computed tomography (CT) images: variation in image reconstruction kernels. These kernels, integral to the Filtered Back-Projection (FBP) process~\cite{schofield2020image}, determine image texture, smoothness, and noise characteristics~\cite{schaller2003spatial}. While the anatomical content of such images remains unchanged, the variation in appearance can cause a severe degradation in segmentation performance of deep neural networks~\cite{choe2019deep,lee2019ct}.

To quantitatively investigate the problem, we collected datasets of paired CT reconstructions~\cite{morozov2020mosmeddata,tsai2021rsna}, i.e., multiple versions of the same scan reconstructed with different kernels. Surprisingly, despite anatomical identity, models trained on one reconstruction style often fail on the other, achieving an average Dice score of only 0.46.

To address this issue, we firstly introduced Filtered Back-Projection Augmentation (FBPAug) -- a novel knowledge-driven augmentation technique rooted in the mathematical model of CT image formation. FBPAug simulates domain shifts by introducing kernel-like transformations directly in the sinogram domain. We derive it by simulating the original FBP process.

FBP consists of two sequential operations: generation of filtered projections and image reconstruction by the Back-Projection (BP) operator. Projections of attenuation map have to be filtered before using them as an input of the Back-Projection operator. The ideal filter in a continuous noiseless case is the ramp filter. Fourier transform of the ramp filter $\kappa(t)$ is $\mathcal{F}[\kappa(t)](w) = |w|$.

The image $I(x, y)$ can be derived as follows:
\begin{equation}
	\label{eqn:main_eq}
	I(x, y) = \text{FBP}(p_\theta(t)) = \text{BP}(p_\theta(t) * \kappa(t)),
\end{equation}
where $*$ is a convolution operator and $t = t(x,y) = x\cos\theta + y\sin\theta$.% and $\kappa(t)$ is the aforementioned ramp filter.

Assume that a set of filtered-projections $p_{\theta}(t)$ available at angles $\theta_1, \theta_2, ..., \theta_n$, such that $\theta_i = \theta_{i - 1} + \Delta\theta,~i=\overline{2,n}$ and $\Delta\theta = \pi / n$. In that case, BP operator transforms a function $f_\theta(t) = f(x\cos\theta + y\sin\theta)$ as follows:
\[
BP(f_\theta(t))(x, y) = \frac{\Delta\theta}{2\pi}\sum\limits_{i=1}^n f_{\theta_i}(x\cos\theta_i + y\sin\theta_i) = \frac{1}{2n}\sum\limits_{i=1}^n f_{\theta_i}(x\cos\theta_i + y\sin\theta_i)
\]

In fact, $\kappa(t)$ that appears in (\ref{eqn:main_eq}) is a generalized function and cannot be expressed as an ordinary function because the integral of $|w|$ in inverse Fourier transform does not converge. However, we utilize the convolution theorem that states that $\mathcal{F}(f*g) = \mathcal{F}(f)\cdot\mathcal{F}(g)$. And after that we can use the fact that the BP operator is a finite weighted sum and Fourier transform is a linear operator as follows:
\[
\mathcal{F}^{-1}\mathcal{F}[I(x, y)] = \mathcal{F}^{-1}\mathcal{F}[\text{BP}(p_\theta * \kappa)] = \text{BP}(\mathcal{F}^{-1}\mathcal{F}[p_\theta * \kappa]) = \text{BP}(\mathcal{F}^{-1}\{\mathcal{F}[p_\theta]\cdot|w|\}),\]
\[I(x, y) = \text{BP}\left(\mathcal{F}^{-1}\{\mathcal{F}[p_\theta]\cdot|w|\}\right).\]

However, in the real world, CT manufacturers use different filters that enhance or weaken the high or low frequencies of the signal. We propose a family of convolution filters $k_{a,b}$ that allows us to obtain a smooth-filtered image given a sharp-filtered image and vice versa. Fourier transform of the proposed filter is expressed as follows:
\[\mathcal{F}[k_{a,b}](w) = \mathcal{F}[\kappa](w)(1 + a \mathcal{F}[\kappa](w)^b) = |w|(1 + a|w|^b).\]

Thus, given a CT image $I$ obtained from a set of projections using one kernel, we can simulate the usage of another kernel as follows:
\[\hat{I}(x, y) = \text{BP}\left(\mathcal{F}^{-1}\{\mathcal{F}[\mathcal{R}(I)]\cdot\mathcal{F}[k_{a,b}]\}\right).\]

Here, $a$ and $b$ are the parameters that influence the sharpness or smoothness of an output image and $\mathcal{R}(I)$ is a Radon transform of image $I$. The output of the Radon transform is a set of projections. Figure~\ref{fig:crops} shows an example of applying the augmentation.

\begin{figure}[h]
	\centering
	\includegraphics[width=0.85\textwidth]{Dissertation/Figures/3_ct/4crops.png}%[width=7cm][width=7cm]
	\caption{An example of paired CT slices and the effect of the augmentation by the proposed method.  The top row contains original images: a slice reconstructed with a smooth (a) and sharp kernel (b). The bottom row shows augmented images: the top-left image processed by FBPAug with parameters $a=30,~b=3$ shifting it from smooth to sharp (c); the top-right image processed by FBPAug with parameters $a=-1,~b=0.7$ from sharp to smooth (d).}
	\label{fig:crops}
\end{figure}

This method produces synthetic images mimicking the effects of alternate kernels and significantly improves segmentation model generalization. In experiments on COVID-19 segmentation, FBPAug increases cross-kernel consistency from 0.46 to 0.76 in Dice score, almost achieving the level of DA methods that require additional data for fine-tuning (Table~\ref{tab:res_final}).


\begin{table}[h]
	\centering
	\caption{Comparison of DA methods in COVID-19 segmentation task. F-/P-Cons stand for F-/P-Consistency. All results are Dice Scores in the format \textit{mean} $\pm$ \textit{std} calculated from $5$-fold cross-validation. We highlight the best scores in every column in \textbf{bold}. \label{tab:res_final}}
	\resizebox{\textwidth}{!}{%
		\begin{tabular}{lccccccc}
			\toprule
			
			& \multirow{2.5}{*}{\textbf{COVID-train}} & \multirow{2.5}{*}{\textbf{COVID-test}} & \multicolumn{5}{c}{\textbf{Paired-Private} \textbf{Consistency}} \\
			\cmidrule(lr){4-8}
			& & & \textbf{FC07/55} & \textbf{FC07/51} & \textbf{SOFT/LUNG} & \textbf{STAND/LUNG}  & \textbf{Mean} \\
			
			
			\midrule
			Baseline & $\mathbf{0.60 \pm 0.04}$ & $0.56 \pm 0.03$ & $0.52 \pm 0.06$ & $0.39 \pm 0.07$ & $0.58 \pm 0.08$ & $0.28 \pm 0.05$ & $0.46 \pm 0.05$\\
			
			\cmidrule{1-8}
			
			FBPAug & $0.59 \pm 0.04$ & $0.62 \pm 0.03$ & $0.80 \pm 0.02$ & $0.71 \pm 0.03$ & $0.85 \pm 0.01$ & $0.65 \pm 0.03$ & $0.76 \pm 0.02$ \\
			
			
			\cmidrule{1-8}
			
			%DANN (Dec) & $.57 \pm 0.04$ & $.61 \pm 0.04$ &  $.61 \pm 0.02$ & $.49 \pm 0.04$ & $.58 \pm 0.03$ & $.31 \pm 0.05$ & $.52 \pm 0.01$ \\
			
			%\cmidrule{1-8}
			
			DANN  & $0.58 \pm 0.05$ & $\mathbf{0.64 \pm 0.02}$ & $0.84 \pm 0.02$ & $0.70 \pm 0.02$ & $\mathbf{0.86 \pm 0.03}$ & $0.66 \pm 0.06$ & $0.78 \pm 0.02$ \\
			
			\cmidrule{1-8}
			
			P-Cons & $0.59 \pm 0.04$ & $0.61 \pm 0.01$ & $0.65 \pm 0.05$ & $0.60 \pm 0.02$ & $0.77 \pm 0.01$ & $0.47 \pm 0.04$ & $0.63 \pm 0.03$\\
			
			\cmidrule{1-8}
			
			
			%F-Cons (Dec) & $\mathbf{.60 \pm 0.03}$ & $.58 \pm 0.02$ & $.62 \pm 0.05$ & $.54 \pm 0.03$ & $.75 \pm 0.01$ & $.40 \pm 0.06$ & $.58 \pm 0.02$ \\
			
			%\cmidrule{1-8}
			
			F-Cons &  $0.57 \pm 0.03$ & $\mathbf{0.64 \pm 0.03}$ & $\mathbf{0.88 \pm 0.01}$ & $\mathbf{0.72 \pm 0.04}$ & $0.83 \pm 0.02$ & $\mathbf{0.70 \pm 0.05}$ & $\mathbf{0.80 \pm 0.01}$ \\
			
			\bottomrule
	\end{tabular}}
\end{table}

Secondly, we propose F-Consistency, a data-driven unsupervised DA technique that uses unlabeled paired reconstructions. By minimizing the mean squared error between intermediate feature maps of the paired images, F-Consistency encourages a domain-invariant representation. The additional optimization objective is formally defined as:

\begin{equation}
	\mathcal{L}_{\text{F-cons}} = \frac{1}{N} \sum_{i=1}^{N} || f_l \left( x_i^A \right) - f_l \left( x_i^B \right) ||_2^2,
\end{equation}

\noindent
where $x_i^A$ and $x_i^B$ are the paired reconstructions within a batch of size $N$, and $f_l \left( \cdot \right)$ denotes the activation at layer $l$. The layer $l$ is chosen based on the SpotTUnet indication, with the latter fine-tuned on FBPAug-synthesized data.

This method further boosts the average consistency to 0.80, outperforming the other state-of-the-art unsupervised DA approaches (Table~\ref{tab:res_final}). In addition to the direct performance benefits, this chapter demonstrates the methodological synergy between SpotTUnet, FBPAug and F-Consistency: the former two enable creation of synthetic data and domain shift-informed network design, while the latter uses this guidance to learn more robust feature spaces.


% %%%%%%%%%%%%%%% CHAPTER 3 %%%%%%%%%%%%%%%


The third chapter, \textbf{Systematically Evaluating DA Methods in 3D Medical Image Segmentation}, presents a large-scale empirical study of domain adaptation techniques in 3D medical image segmentation. While domain shift is known to severely degrade model performance in these tasks, existing DA methods are often evaluated under overly narrow conditions. To address this gap, the chapter introduces a comprehensive benchmark, M3DA, and a novel Burdenko’s Glioblastoma Progression (BGP) dataset, specifically designed for evaluating the robustness and scalability of unsupervised DA methods under diverse and clinically relevant conditions.

The benchmark consists of four publicly available multiclass segmentation datasets~\cite{amos,brats,cc359,lidc} and includes eight domain shift scenarios (Figure~\ref{fig:teaser2}). Each scenario is framed as an unsupervised domain adaptation problem formalized as follows:

\begin{figure*}[h]
	\centering
	\includegraphics[width=1\linewidth]{Dissertation/Figures/4_da_bench/fig2_bench_examples.png}
	\caption{Examples from individual domains in M3DA without segmentation masks for visual comparison between domains. Left to right, top to bottom: CT to MR, CT to LDCT, CT CE to CT native, CE T1 to T1, T1 Field (1.5T to 3T), T1 Scanner (Philips to Siemens).}
	\label{fig:teaser2}
\end{figure*}

First, we consider a semantic segmentation problem of 3D medical images, which we call a downstream task. Any downstream model works with input samples $x \in X$ and the corresponding segmentation masks $y \in Y$, where $X$ and $Y$ are some input image and label spaces. If $x \in \R^{H \times W \times D}$, segmentation mask is of the same spatial size $y \in \R^{H \times W \times D}$, where every element belongs to a predefined set of labels $y^{(h,w,d)} \in \{ 0, 1, \dots, C \}$, $0$ is background and $C$ is the number of foreground classes.

Following the standard problem setting~\cite{dann}, we assume that two distributions $\mathcal{S}(x, y)$ and $\mathcal{T}(x, y)$ exist on $X \otimes Y$, called \textit{source} and \textit{target} distributions. At the training time, we have a set of source training samples $X^s = \{ x_i^s \}_{i=1}^n$ with the corresponding masks $Y^s = \{ y_i^s \}_{i=1}^n$ and a set of target training samples $X^t_{tr}$ without annotations; source images and masks are considered to be sampled from $\mathcal{S}$, $(x_i^s, y_i^s) \sim \mathcal{S}(x, y)$. Our goal is to predict segmentations $y$ given the input from the marginal distribution of target images, $x \sim \mathcal{T}(x)$. To evaluate algorithms, we have target testing samples $X^t_{ts}$ with masks $Y^t_{ts}$ available only for evaluation purposes.

To assess the current state of the field, we evaluated over ten state-of-the-art DA techniques, including pixel-level generative, feature-level adversarial, and self-supervised approaches. Despite their apparent effectiveness in some scenarios, the results reveal that existing methods often fail to generalize beyond specific setups (Table~\ref{tab:metrics_pure}). For instance, the best-performing method (supplemented with the extensive data augmentation) reduces the performance drop caused by domain shift by only 62\% on average, leaving a significant residual gap (Table~\ref{tab:ablation_aug}). This finding underscores the need for robust and scalable DA strategies tailored to the unique challenges of 3D medical data.

\begin{table}
	\centering
	\caption{Main results on M3DA benchmark in terms of multiclass average Dice score, where background label is excluded from quantification. The best results in each column are highlighted in \textbf{bold}. Case-wise standard deviations for these experiments are provided in parentheses.} %Foundational models UniModel and SAM-Med3D were finetuned in the baseline setting, similar to U-Net.}
	
	\resizebox{\textwidth}{!}{%
		\begin{tabular}{lcccccccccc}
			\toprule
			& MR$\rightarrow$CT & CT$\rightarrow$MR & CT$\rightarrow$LDCT & CE CT$\rightarrow$CT & T1 CE$\rightarrow$T1 & T1 F & T1 Sc & T1 Mix & \textit{avg DSC} & \textit{avg gap} \\
			
			\midrule
			
			U-Net (Baseline)      & 0.032 (0.045) & 0.032 (0.038) & 0.133 (0.162) & 0.228 (0.265) & 0.426 (0.172) & 0.741 (0.067) & 0.766 (0.025) & 0.560 (0.159) & 0.365 & 0.0\% \\
			
			SAM-Med3D \cite{sammed} & 0.019 (0.013) & 0.037 (0.031) & 0.524 (0.120) & 0.412 (0.307) & 0.270 (0.155) & 0.645 (0.080) & 0.758 (0.035) & 0.486 (0.274) & 0.394 & -1.0\% \\
			
			
			UniModel \cite{unimodel} & 0.027 (0.017) & 0.012 (0.013) & 0.252 (0.191) & \textbf{0.470 (0.313)} & 0.331 (0.143) & 0.740 (0.064) & 0.736 (0.038) & 0.618 (0.179) & 0.398 & 7.4\%\\
			
			\midrule
			
			% HM & 0.331 (0.182) & 0.222 (0.128) & 0.111 (0.174) & 0.133 (0.221) & 0.341 (0.183) & 0.789 (0.069) & 0.748 (0.090) & 0.504 (0.195) & 0.397 & -1.1\% \\
			
			CycleGAN 3D \cite{cyclegan3d} & 0.333 (0.128) & 0.264 (0.113) & 0.326 (0.175) & 0.130 (0.203) & 0.345 (0.175) & 0.791 (0.035) & 0.713 (0.023) & 0.762 (0.017) & 0.458 & 9.5\% \\ % \cite{cyclegan3d}
			
			MinEnt \cite{entropy} & 0.140 (0.136) & 0.172 (0.149) & 0.505 (0.127) & 0.392 (0.323) & 0.429 (0.168) & 0.770 (0.038) & 0.798 (0.025) & 0.776 (0.088) & 0.498 & 28.5\% \\
			
			CycleGAN 2D \cite{cyclegan} & 0.205 (0.153) & 0.406 (0.144) & 0.530 (0.187) & 0.216 (0.260) & 0.398 (0.181) & 0.852 (0.015) & 0.801 (0.027) & 0.795 (0.024) & 0.525 & 30.2\% \\ % \cite{cyclegan}
			
			GIN \cite{gin} & \textbf{0.589 (0.144)} & \textbf{0.637 (0.105)} & 0.722 (0.108) & 0.163 (0.238) & 0.382 (0.181) & 0.837 (0.066) & 0.709 (0.069) & 0.804 (0.062) & 0.605 & 33.6\% \\
			
			AdaBN \cite{adabn} & 0.322 (0.157) & 0.353 (0.177) & 0.587 (0.202) & 0.295 (0.291) & 0.433 (0.165) & 0.778 (0.042) & 0.833 (0.020) & 0.796 (0.059) & 0.550 & 35.0\% \\
			
			DANN \cite{dann_medim} & 0.296 (0.147) & 0.278 (0.135) & 0.699 (0.148) & 0.409 (0.297) & 0.416 (0.161) & 0.730 (0.078) & 0.833 (0.029) & 0.776 (0.082) & 0.555 & 36.2\% \\
			
			IN \cite{instance_norm} & 0.303 (0.149) & 0.308 (0.143) & 0.668 (0.167) & 0.427 (0.287) & 0.428 (0.155) & 0.756 (0.058) & 0.838 (0.025) & 0.784 (0.078) & 0.564 & 39.6\% \\ % \cite{instance_norm}
			
			MIND \cite{dg_tta} & 0.560 (0.171) & 0.588 (0.125) & 0.237 (0.148) & 0.425 (0.236) & 0.335 (0.162) & 0.865 (0.035) & 0.869 (0.039) & 0.845 (0.033) & 0.590 & 45.9\% \\
			
			Gamma augm     & 0.349 (0.182) & 0.166 (0.146) & 0.241 (0.230) & 0.441 (0.313) & 0.443 (0.163) & 0.893 (0.018) & \textbf{0.910 (0.006)} & 0.910 (0.012) & 0.544 & 48.3\% \\
			
			Self-Ensemble \cite{se_medim} & 0.391 (0.133) & 0.388 (0.101) & 0.603 (0.189) & 0.332 (0.291) & 0.388 (0.175) & 0.906 (0.023) & 0.893 (0.013) & \textbf{0.918 (0.018)} & 0.602 & 51.7\% \\
			
			nnAugm \cite{nnunet} & 0.166 (0.125) & 0.102 (0.090) & \textbf{0.779 (0.103)} & 0.392 (0.315) & \textbf{0.446 (0.164)} & \textbf{0.910 (0.010)} & 0.897 (0.012) & 0.889 (0.012) & 0.573 & 51.9\% \\
			
			% nnUNet          & 0.397 & 0.355 & 0.750 & 0.373 & 0.330 & 0.923 & 0.914 & 0.907 & 0.619 & 54.9\% \\
			
			% \midrule
			
			% best in setup & 0.589 & 0.637 & 0.779 & 0.470 & 0.446 & 0.910 & 0.910 & 0.918 & 0.704 & 72.9\% \\
			
			\midrule
			
			U-Net (Oracle)        & 0.842 (0.092) & 0.826 (0.035) & 0.814 (0.095) & 0.519 (0.297) & 0.686 (0.178) & 0.954 (0.017) & 0.957 (0.012) & 0.958 (0.009) & 0.820 & 100.0\% \\
			
			\bottomrule
			
	\end{tabular}}
	\label{tab:metrics_pure}
\end{table}


\begin{table}[ht]
	\centering
	\caption{Performance comparison of different methods supplemented with nnU-Net augmentations. Colored numbers show an improvement (or a decline, respectively) over a non-augmented method. GIN and MIND were only trained with nnU-Net augmentations.}
	
	% Add these color definitions to your preamble
	\definecolor{darkGreen}{RGB}{0, 102, 0}     % For improvements > 0.15
	\definecolor{medGreen}{RGB}{0, 153, 0}      % For improvements 0.05 to 0.15
	\definecolor{lightGreen}{RGB}{144, 238, 144} % For small improvements 0 to 0.05
	\definecolor{lightRed}{RGB}{255, 200, 200}   % For negative values
	
	\resizebox{\textwidth}{!}{%
		\begin{tabular}{lcccccccccc}
			\toprule
			& MR$\rightarrow$CT & CT$\rightarrow$MR & CT$\rightarrow$LDCT & CE CT$\rightarrow$CT & T1 CE$\rightarrow$T1 & T1 F & T1 Sc & T1 Mix & \textbf{avg DSC} & \textbf{avg gap} \\
			
			\midrule
			
			% GIN & 0.589 & 0.637 & 0.722 & 0.163 & 0.382 & 0.837 & 0.709 & 0.804 & 0.605 & 33.6\% \\
			
			CycleGAN 3D & 0.364 \textcolor{lightGreen}{$\uparrow$0.031} & 0.464 \textcolor{darkGreen}{$\uparrow$0.200} & 0.679 \textcolor{darkGreen}{$\uparrow$0.353} & 0.221 \textcolor{medGreen}{$\uparrow$0.091} & 0.379 \textcolor{lightGreen}{$\uparrow$0.034} & 0.825 \textcolor{lightGreen}{$\uparrow$0.034} & 0.810 \textcolor{medGreen}{$\uparrow$0.097} & 0.779 \textcolor{lightGreen}{$\uparrow$0.017} & 0.565 \textcolor{medGreen}{$\uparrow$0.107} & 34.1\% \textcolor{darkGreen}{$\uparrow$24.6\%} \\
			CycleGAN 2D & 0.301 \textcolor{medGreen}{$\uparrow$0.096} & 0.461 \textcolor{medGreen}{$\uparrow$0.055} & 0.666 \textcolor{medGreen}{$\uparrow$0.136} & 0.333 \textcolor{medGreen}{$\uparrow$0.117} & 0.416 \textcolor{lightGreen}{$\uparrow$0.018} & 0.865 \textcolor{lightGreen}{$\uparrow$0.013} & 0.850 \textcolor{lightGreen}{$\uparrow$0.049} & 0.815 \textcolor{lightGreen}{$\uparrow$0.020} & 0.588 \textcolor{medGreen}{$\uparrow$0.063} & 45.5\% \textcolor{medGreen}{$\uparrow$15.3\%} \\
			
			% MIND & 0.560 & 0.588 & 0.237 & 0.425 & 0.335 & 0.865 & 0.869 & 0.845 & 0.590 & 45.9\% \\
			
			Baseline (nnAugm) & 0.166 \textcolor{medGreen}{$\uparrow$0.134} & 0.102 \textcolor{medGreen}{$\uparrow$0.070} & 0.779 \textcolor{darkGreen}{$\uparrow$0.646} & 0.392 \textcolor{darkGreen}{$\uparrow$0.164} & 0.446 \textcolor{lightGreen}{$\uparrow$0.020} & 0.910 \textcolor{darkGreen}{$\uparrow$0.169} & 0.897 \textcolor{medGreen}{$\uparrow$0.131} & 0.889 \textcolor{darkGreen}{$\uparrow$0.329} & 0.573 \textcolor{darkGreen}{$\uparrow$0.208} & 51.9\% \textcolor{darkGreen}{$\uparrow$51.9\%} \\  % nnAugm (Baseline)
			DANN        & 0.414 \textcolor{medGreen}{$\uparrow$0.118} & 0.349 \textcolor{medGreen}{$\uparrow$0.071} & 0.809 \textcolor{medGreen}{$\uparrow$0.110} & 0.411 \textcolor{lightGreen}{$\uparrow$0.002} & 0.403 \textcolor{lightRed}{$\downarrow$-0.013} & 0.899 \textcolor{darkGreen}{$\uparrow$0.169} & 0.848 \textcolor{lightGreen}{$\uparrow$0.015} & 0.885 \textcolor{medGreen}{$\uparrow$0.109} & 0.627 \textcolor{medGreen}{$\uparrow$0.072} & 54.9\% \textcolor{darkGreen}{$\uparrow$23.3\%} \\
			
			IN          & 0.422 \textcolor{medGreen}{$\uparrow$0.119} & 0.471 \textcolor{darkGreen}{$\uparrow$0.163} & 0.796 \textcolor{medGreen}{$\uparrow$0.128} & 0.410 \textcolor{lightRed}{$\downarrow$-0.017} & 0.416 \textcolor{lightRed}{$\downarrow$-0.012} & 0.907 \textcolor{darkGreen}{$\uparrow$0.151} & 0.854 \textcolor{lightGreen}{$\uparrow$0.016} & 0.883 \textcolor{medGreen}{$\uparrow$0.099} & 0.645 \textcolor{medGreen}{$\uparrow$0.081} & 58.1\% \textcolor{darkGreen}{$\uparrow$26.6\%} \\
			AdaBN       & 0.495 \textcolor{darkGreen}{$\uparrow$0.173} & 0.532 \textcolor{darkGreen}{$\uparrow$0.179} & 0.604 \textcolor{lightGreen}{$\uparrow$0.017} & 0.365 \textcolor{medGreen}{$\uparrow$0.070} & 0.454 \textcolor{lightGreen}{$\uparrow$0.021} & 0.907 \textcolor{medGreen}{$\uparrow$0.129} & 0.890 \textcolor{medGreen}{$\uparrow$0.057} & 0.892 \textcolor{medGreen}{$\uparrow$0.096} & 0.642 \textcolor{medGreen}{$\uparrow$0.092} & 59.2\% \textcolor{darkGreen}{$\uparrow$24.2\%} \\
			
			SE          & 0.459 \textcolor{medGreen}{$\uparrow$0.068} & 0.571 \textcolor{darkGreen}{$\uparrow$0.183} & 0.768 \textcolor{darkGreen}{$\uparrow$0.165} & 0.389 \textcolor{medGreen}{$\uparrow$0.057} & 0.374 \textcolor{lightRed}{$\downarrow$-0.014} & 0.902 \textcolor{lightRed}{$\downarrow$-0.004} & 0.907 \textcolor{lightGreen}{$\uparrow$0.014} & 0.888 \textcolor{lightRed}{$\downarrow$-0.030} & 0.657 \textcolor{medGreen}{$\uparrow$0.055} & 60.1\% \enspace \textcolor{medGreen}{$\uparrow$8.4\%} \\
			MinEnt      & 0.388 \textcolor{darkGreen}{$\uparrow$0.248} & 0.362 \textcolor{darkGreen}{$\uparrow$0.190} & 0.788 \textcolor{darkGreen}{$\uparrow$0.283} & 0.449 \textcolor{medGreen}{$\uparrow$0.057} & 0.448 \textcolor{lightGreen}{$\uparrow$0.019} & 0.903 \textcolor{medGreen}{$\uparrow$0.133} & 0.901 \textcolor{medGreen}{$\uparrow$0.103} & 0.892 \textcolor{medGreen}{$\uparrow$0.116} & 0.641 \textcolor{medGreen}{$\uparrow$0.143} & 62.0\% \textcolor{darkGreen}{$\uparrow$33.5\%} \\
			
			\midrule
			% \textbf{Average improvement} & 0.376 \textcolor{medGreen}{$\uparrow$0.123} & 0.414 \textcolor{darkGreen}{$\uparrow$0.139} & 0.736 \textcolor{darkGreen}{$\uparrow$0.230} & 0.371 \textcolor{medGreen}{$\uparrow$0.068} & 0.417 \textcolor{lightGreen}{$\uparrow$0.009} & 0.890 \textcolor{medGreen}{$\uparrow$0.099} & 0.870 \textcolor{medGreen}{$\uparrow$0.060} & 0.865 \textcolor{medGreen}{$\uparrow$0.095} & 0.617 \textcolor{medGreen}{$\uparrow$0.103} \\
			\textbf{average} &\textcolor{medGreen}{$\uparrow$0.123} & \textcolor{darkGreen}{$\uparrow$0.139} & \textcolor{darkGreen}{$\uparrow$0.230} & \textcolor{medGreen}{$\uparrow$0.068} & \textcolor{lightGreen}{$\uparrow$0.009} & \textcolor{medGreen}{$\uparrow$0.099} & \textcolor{medGreen}{$\uparrow$0.060} & \textcolor{medGreen}{$\uparrow$0.095} & \textcolor{medGreen}{$\uparrow$0.103} & \textcolor{darkGreen}{$\uparrow$26.0\%} \\  % 53.2\%
			
			\bottomrule
	\end{tabular}}
	\label{tab:ablation_aug}
\end{table}

To support research under highly realistic clinical conditions, the chapter also presents the Burdenko’s Glioblastoma Progression (BGP) dataset. The BGP dataset contains MRI studies from 180 glioblastoma patients acquired for radiotherapy planning, with images sourced from four different scanner vendors and spanning diverse acquisition protocols. This allows researchers to test the generalization of models under in-the-wild domain shifts where the target domain is not fully known or controlled.

In summary, this chapter contributes a rigorous benchmarking suite for DA in 3D medical imaging and reveals key limitations of current approaches. It motivates future work on methods that can adapt effectively across multiple axes of variability without relying on restrictive assumptions about the target domain.


% %%%%%%%%%%%%%%% CHAPTER 4 %%%%%%%%%%%%%%%


The fourth chapter, \textbf{Benchmark for OOD in 3D Medical Image Segmentation}, addresses the critical challenge of out-of-distribution (OOD) detection in clinical deployment of segmentation models. While previous chapters focused on adapting models to known domain shifts, this chapter explores the complementary task of identifying when input data deviates significantly from the training distribution -- a prerequisite for safe model application in real-world medical environments.

To systematically investigate this problem, the chapter introduces a dedicated benchmark for OOD detection in 3D medical image segmentation. The benchmark is designed around multiple clinically relevant use cases, where segmentation models are applied to samples that violate assumptions about modality, anatomy, or acquisition settings. These include acquisition protocol, patient population, and anatomical region changes and imaging artifact encounter in both CT (Figure~\ref{fig:ct}) and MRI (Figure~\ref{fig:mri}) modalities.

\begin{figure}[h]
	\centering
	\includegraphics[width=\textwidth]{Dissertation/Figures/5_ood_bench/ct_examples_2.pdf}
	\caption{Examples of CT images (representative axial slices) from different simulated OOD sources in our benchmark.}
	\label{fig:ct}
\end{figure}

\begin{figure}[h]
	\centering
	\includegraphics[width=\textwidth]{Dissertation/Figures/5_ood_bench/mri_examples_2.pdf}
	\caption{Examples of MRI images (representative axial slices) from different simulated OOD sources in our benchmark.}
	\label{fig:mri}
\end{figure}

The chapter evaluates several state-of-the-art OOD detection methods, including softmax-based confidence scores, Monte-Carlo dropout, and self-supervised approaches. Despite their success in 2D natural image tasks, most methods exhibit poor performance in the 3D medical setting, with high false positive rates (Table~\ref{tab:res_fpr}).

\begin{table}[h]
	\centering
	\caption{Comparison of the considered OOD detection methods in terms of FPR@TPR95\% scores (lower is better). We highlight the best scores in every row in \textbf{bold} and ranked the methods by their average performance. The first and second sections correspond to CT and MRI setups, respectively.}
	\resizebox{\textwidth}{!}{%
		\begin{tabular}{llllllllll}
			\toprule
			\textbf{OOD Setup} &            \textbf{IHF-NN} &             \textbf{SVD} &             \textbf{IHF-Mah} &      \textbf{MOOD-1} &           \textbf{G-ODIN} &             \textbf{Volume} &   \textbf{MCD} & \textbf{Ensemble} &   \textbf{Entropy} \\
			\midrule
			Location (Head)           &  \textbf{0.00}  &  \textbf{0.00}  &  \textbf{0.00}  &            0.12 &            0.55 &            0.53 &  0.36 &     0.51 &  0.56 \\
			Location (Liver)          &            0.51 &  \textbf{0.13}  &            0.64 &            0.56 &            0.56 &            0.84 &  0.89 &     0.93 &  0.78 \\
			Population (COVID-19)     &            0.54 &            0.75 &            0.72 &  \textbf{0.51}  &            0.54 &            0.82 &  0.58 &     0.58 &  0.87 \\
			Scanner                 &            0.88 &            0.89 &            0.85 &  \textbf{0.73}  &            0.92 &            0.86 &  0.89 &     0.90 &  0.83 \\
			Synthetic (Elastic)       &  \textbf{0.15}  &            0.37 &            0.67 &            0.16 &            0.59 &            0.81 &  0.42 &     0.37 &  0.84 \\
			Synthetic (Image noise)  &            0.49 &            0.37 &            0.62 &  \textbf{0.11}  &            0.89 &            0.85 &  0.87 &     0.82 &  0.81 \\
			\midrule
			Population (Glioblastoma) &  \textbf{0.00}  &  \textbf{0.00}  &  \textbf{0.00}  &            0.10 &            0.21 &            0.01 &  0.85 &     0.81 &  0.86 \\
			Population (Healthy)      &  \textbf{0.00}  &  \textbf{0.00}  &  \textbf{0.00}  &            0.11 &  \textbf{0.00}  &  \textbf{0.00}  &  0.88 &     10.0 &  0.85 \\
			Scanner                   &  \textbf{0.00}  &  \textbf{0.00}  &  \textbf{0.00}  &            0.15 &  \textbf{0.00}  &            0.74 &  0.63 &     0.66 &  0.89 \\
			Synthetic (K-space noise) &  \textbf{0.00}  &            0.36 &  \textbf{0.00}  &            0.88 &            0.88 &            0.90 &  0.82 &     0.77 &  0.73 \\
			Synthetic (Anisotropy)    &            0.09 &            0.20 &  \textbf{0.05}  &            0.57 &            0.88 &            0.93 &  0.77 &     0.77 &  0.81 \\
			Synthetic (Motion)        &  \textbf{0.00}  &            0.58 &  \textbf{0.00}  &            0.73 &            0.93 &            0.94 &  0.85 &     0.88 &  0.91 \\
			Synthetic (Image noise)   &            0.47 &            0.33 &            0.47 &  \textbf{0.30}  &            0.56 &            0.71 &  0.78 &     0.75 &  0.75 \\
			\midrule
			CT average                &            0.43 &            0.42 &            0.58 &  \textbf{0.36}  &            0.67 &            0.79 &  0.67 &     0.68 &  0.78 \\
			MRI average               &            0.08 &            0.21 &  \textbf{0.07}  &            0.41 &            0.50 &            0.60 &  0.80 &     0.81 &  0.83 \\
			\bottomrule
	\end{tabular}}
	\label{tab:res_fpr}
\end{table}

To complement these baselines, the chapter proposes a novel lightweight method, Intensity Histogram Features (IHF), which captures global image statistics using simple summary histograms of voxel intensities. As SpotTUnet-based analysis suggests, the earlier network's layers contain the most domain-specific information. Taking this analysis to the extreme, we hypothesize that we can extract enough domain-specific information directly from the image (i.e., the zeroth network's layer). A histogram is a convenient way to do so.

We schematically present our method, called Intensity Histogram Features (IHF), in Figure~\ref{fig:ihf}. It consists of three steps: (1) calculating intensity histograms of images and using them as vectors, (2) reducing their dimensionality with PCA, and (3) running an outlier detection algorithm on these vectors.

\begin{figure}[h]
	\centering
	\includegraphics[width=\textwidth]{Dissertation/Figures/5_ood_bench/method-1.pdf}
	\caption{The proposed OOD detection method, called Intensity Histogram Features (IHF). It consists of three steps: calculating a $m$-dimensional vector as a histogram bin values from the preprocessed image (Step 1), fitting and applying PCA to the occuring data, and calculating Mahalanobis distance between a test vector and ID samples distribution (Step 3). We apply IHF to the 3D images and illustrate the process using 2D axial slices for simplicity. (* PCA is fitted once on all training data.)}
	\label{fig:ihf}
\end{figure}

Step 1: preprocessing and histograms. All images undergo the same preprocessing pipeline to standardize the intensity distribution. Given a preprocessed image $x$, we compute a probability density function of its intensities in $m$ bins, a histogram $e(x) \in \mathbb{R}^m$, and further use these vectors $e(x)$.
	
Step 2: Principal Component Analysis (PCA). As an optional step, we use PCA~\cite{pca} to reduce the dimensions $m$. The main reason to use it is that some outlier detection algorithms at Step 3 behave unstable in high dimensional spaces. For instance, calculating Mahalanobis distance requires reversing the empirical sample covariance matrix, and this matrix is likely to become ill-conditioned or singular with larger $m$. We fit PCA$_v$ once on the training data $E_{tr}$ to preserve $v = 99.99\%$ of the explained variance. This way, we eliminate the potential instability and preserve the distribution properties. $E_{tr}$ consists of row-vectors $e(x_{tr})$ for all training images $x_{tr} \in X_{tr}$. Further, we use transformed vectors $\tilde{e}(x) = \text{PCA}_v (e(x))$.
	
Step 3: OOD detection algorithm. To calculate an OOD score for $x$, we can apply any distance- or density-based outlier detection method. As our first approach, we can calculate Mahalanobis distance $S_{Mah}(x)$:

\begin{equation}
	\label{eq:mah}
	S_{Mah}(x) = \sqrt{ \left( \tilde{e}(x) - \hat{\mu} \right)^T \hat{\Sigma}^{-1} \left( \tilde{e}(x) - \hat{\mu} \right) },
\end{equation}

\noindent
where $\hat{\mu}$ and $\hat{\Sigma}$ are the estimated mean and covariance matrix on the training set, $\hat{\mu} = \frac{1}{|X_{tr}|} \sum_{x_{tr} \in X_{tr}} \tilde{e} \left(x_{tr}\right)$ and $\hat{\Sigma} = \frac{1}{|X_{tr}|} \sum_{x_{tr} \in X_{tr}} \left( \tilde{e} (x_{tr}) - \hat{\mu} \right) \left( \tilde{e} (x_{tr}) - \hat{\mu} \right)^T$.

Alternatively, one can calculate the distance to the nearest neighbor (min-distance) $S_{NN}(x)$:

\begin{equation}
	\label{eq:nn}
	S_{NN}(x) = \min_{x_{tr} \in X_{tr}} || \tilde{e} (x) - \tilde{e} (x_{tr}) ||_2.
\end{equation}

Using $S_{Mah}$ (Equation~\ref{eq:mah}) and $S_{NN}$ (Equation~\ref{eq:nn}) corresponds to the methods IHF-Mah and IHF-NN, respectively. We included them in our experiments independently.

Despite its simplicity, IHF achieves competitive performance: ranking top-1 in multiple benchmark scenarios (Table~\ref{tab:res_fpr}) and ranked the top-2 solution in the Medical Out-of-Distribution Challenge (MOOD) in both 2022 and 2023~\cite{zimmerer2022mood}. The method’s efficiency and interpretability make it a strong candidate for integration into clinical pipelines.

In summary, this chapter reveals the limitations of current OOD detection methods in 3D medical image segmentation, provides a standardized evaluation framework and a strong OOD detection baseline for future research. Together with previous chapters, it contributes to the broader goal of building safe, adaptive, and deployable AI systems for medical imaging.


% %%%%%%%%%%%%%%% CONCLUSION %%%%%%%%%%%%%%

The last chapter, \textbf{Conclusion}, summarizes the results and discusses future work.

% The \textbf{conclusions chapter} summarizes the thesis, consolidates the presented results, and offers a concise summary of the investigation’s significance

\section*{Conclusion}

This dissertation is devoted to the development of mathematically grounded and practically robust deep learning methods for domain adaptation and out-of-distribution detection in 3D medical image segmentation. The work systematically addresses the challenge of domain shift, which remains a critical obstacle to the reliable deployment of machine learning systems in clinical environments.

\begin{enumerate}
	
	\item A gradient-based supervised domain adaptation method, SpotTUnet, was proposed to address the problem of layer-wise adaptation in convolutional neural networks for medical image segmentation. The method optimizes a learnable layer selection policy, allowing to fine-tune the most shift-sensitive layers. The effectiveness of this approach was confirmed through extensive experimental validation, including statistically significant improvements over common fine-tuning baselines in few-shot adaptation scenarios. From a practical standpoint, SpotTUnet provides interpretable visualizations of domain shift sensitivity across network layers, which we further used to guide the design of F-Consistency and to enhance performance of existing DA algorithms, such as DANN~\cite{dann}, on the M3DA benchmark. Moreover, conclusions drawn from SpotTUnet contributed to the development of the IHF method for out-of-distribution detection. The proposed approach can be further extended to unsupervised setting by leveraging auxiliary domain shift criteria and emerging transformer-based architectures~\cite{unetr}.
        
    \item Two complementary DA methods were proposed to address performance degradation in CT segmentation caused by variations in reconstruction kernels. First, a knowledge-driven augmentation technique, Filtered Back-Projection Augmentation (FBPAug), was developed to simulate such kernel-induced domain shifts by modeling the mathematics of CT image formation in sinogram space. Second, a data-driven unsupervised DA method, F-Consistency, was introduced to align internal network representations of paired CT images reconstructed with different kernels. Both methods demonstrated statistically significant improvements over state-of-the-art approaches: FBPAug achieved high consistency in the zero-shot adaptation setting, while F-Consistency further increased segmentation quality in the standard unsupervised DA setting. FBPAug has been integrated into the production pipeline of a medical imaging startup, where it significantly improved the robustness and accuracy of CT-based segmentation models used in clinical practice. Further research may focus on combining FBPAug-generated pseudo-pairs with the F-Consistency objective in a self-supervised learning framework to obtain even more robust pretrained foundational model.
    
    \item A large-scale benchmark, M3DA, was developed to evaluate unsupervised DA methods in 3D medical image segmentation under realistic and diverse shift scenarios. The benchmark reveals that existing methods close only 62\% of the performance gap on average. To support clinical relevance, a new multi-modal dataset, BGP, was published for glioblastoma segmentation with high intra-institution variability. These contributions enable systematic comparison of methods and guide the design of robust and generalizable DA algorithms.
    
    \item A benchmark for OOD detection in 3D medical segmentation was constructed, including clinically relevant scenarios, and was used to identify fundamental limitations of current methods. A lightweight method, IHF, was proposed as an interpretable and computationally efficient baseline, achieving top-2 in the MOOD 2022 and 2023 challenges. The benchmark and IHF serve as a foundation for further theoretical study of distributional uncertainty in medical imaging data.
    
\end{enumerate}

The results and insights of this dissertation have already served as a foundation for several subsequent scientific works. Firstly, we used the proposed OOD benchmark to develop a novel OOD detection metric and theoretically justified framework~\cite{vasiliuk2023redesigning}. We also used components of this benchmark to analyze predictive uncertainty at the level of distinct predicted components~\cite{vasiliuk2022exploring}. Finally, the authors of~\cite{goncharov2023vox2vec}, integrated FBPAug into a self-supervised learning framework, realizing a direction for augmentation that was originally suggested in this dissertation. These works demonstrate the relevance of the presented contributions and point toward continued progress in the robust deployment of deep learning in medical imaging.



\section*{\bibtitleauthorEn}
% Publications on the dissertation contents
%Author's publications on the dissertation topic


\begin{enumerate}
	
	\item \textbf{Shirokikh B.}, Zakazov I., Chernyavskiy A., Fedulova I., Belyaev M. First U-Net layers contain more domain specific information than the last ones // Lecture Notes in Computer Science (including subseries Lecture Notes in Artificial Intelligence and Lecture Notes in Bioinformatics). -- 2020. -- Vol. 12444. -- p. 117--126. -- DOI: 10.1007/978-3-030-60548-3\_12 (Indexed in Scopus);
	
    \item Saparov T., Kurmukov A., \textbf{Shirokikh B.}, Belyaev M. Zero-Shot Domain Adaptation in CT Segmentation by Filtered Back Projection Augmentation // Lecture Notes in Computer Science (including subseries Lecture Notes in Artificial Intelligence and Lecture Notes in Bioinformatics). -- 2021. -- Vol. 13003. -- p. 243--250. -- DOI: 10.1007/978-3-030-88210-5\_24 (Indexed in Scopus);
    
    \item Goncharov M.\footnotemark[1], Pisov M.\footnotemark[1], Shevtsov A.\footnotemark[1], \textbf{Shirokikh B.}\footnotemark[1], Kurmukov A., Blokhin I., Chernina V., Solovev A., Gombolevskiy V., Morozov S., Belyaev M. CT-Based COVID-19 triage: Deep multitask learning improves joint identification and severity quantification // Medical Image Analysis. -- 2021. -- Vol. 71. -- p. 102054. -- DOI: 10.1016/j.media.2021.102054 \footnotetext[1]{shared contribution} (Indexed in Scopus, Q1 journal, IF: 10.7);
    
    \item  Zakazov I.\footnotemark[2], \textbf{Shirokikh B.}\footnotemark[2], Chernyavskiy A., Belyaev M. Anatomy of domain shift impact on U-Net layers in MRI segmentation // Lecture Notes in Computer Science (including subseries Lecture Notes in Artificial Intelligence and Lecture Notes in Bioinformatics). -- 2021. -- Vol. 12903. -- p. 211--220. -- DOI: 10.1007/978-3-030-87199-4\_20 \footnotetext[2]{shared contribution} (Indexed in Scopus, proceedings of A conference);
    
    \item Shimovolos S., Shushko A., Belyaev M., \textbf{Shirokikh B.} Adaptation to CT Reconstruction Kernels by Enforcing Cross-Domain Feature Maps Consistency // Journal of Imaging. -- 2022. -- Vol. 8, Is. 9. -- p. 234. -- DOI: 10.3390/jimaging8090234 (Indexed in Scopus, Q2 journal, IF: 2.7);
    
    \item \textbf{Shirokikh B.}, Dalechina A., Shevtsov A., Krivov E., Kostjuchenko V., Durgaryan A., Galkin M., Golanov A., Belyaev M. Systematic clinical evaluation of a deep learning method for medical image segmentation: radiosurgery application // IEEE Journal of Biomedical and Health Informatics. -- 2022. -- Vol. 26, Is. 7. -- p. 3037--3046. -- DOI: 10.1109/JBHI.2022.3153394 (Indexed in Scopus, Q1 journal, IF: 6.7);
    
    \item Vasiliuk A., Frolova D., Belyaev M., \textbf{Shirokikh B.} Limitations of out-of-distribution detection in 3d medical image segmentation // Journal of Imaging. -- 2023. -- Vol. 9, Is. 9. -- p. 191. -- DOI: 10.3390/jimaging9090191 (Indexed in Scopus, Q2 journal, IF: 2.7);
    
    \item Certificate of state registration of computer programs No. 2021612647(RU), Belyaev M., Pisov M., \textbf{Shirokikh B.}, 20 Feb 2021 (in Russian).
        
\end{enumerate}

    


\printbibliography[heading=subbibliography]

\ifnumequal{\value{contnumfig}}{1}{\counterwithout{figure}{chapter}}{\counterwithin{figure}{chapter}}
\ifnumequal{\value{contnumtab}}{1}{\counterwithout{table}{chapter}}{\counterwithin{table}{chapter}}


% \input{Synopsis/content}      % Содержание автореферата

%%% Выходные сведения типографии
\newpage\thispagestyle{empty}

\vspace*{0pt plus1fill}

\small
\begin{center}
    \textit{\thesisAuthor}
    \par\medskip

    \thesisTitle
    \par\medskip

    Автореф. дис. на соискание ученой степени \thesisDegreeShort
    \par\bigskip

    Подписано в печать \blank[\widthof{999}].\blank[\widthof{999}].\blank[\widthof{99999}].
    Заказ № \blank[\widthof{999999999999}]

    Формат 60\(\times\)90/16. Усл. печ. л. 1. Тираж 50 экз.
    %Это не совсем формат А5, но наиболее близкий, подробнее: http://ru.wikipedia.org/w/index.php?oldid=78976454

    Типография \blank[0.5\linewidth]
\end{center}
\cleardoublepage

\end{document}
